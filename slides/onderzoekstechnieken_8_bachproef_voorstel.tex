%%----------------------------------------------------------------------------
%% Presentatie HoGent Bedrijf en Organisatie
%%----------------------------------------------------------------------------
%% Auteur: Bert Van Vreckem [bert.vanvreckem@hogent.be]

\documentclass[usenames,dvipsnames]{beamer}

%==============================================================================
% Aanloop
%==============================================================================

%---------- Packages ----------------------------------------------------------
\usepackage{etex}
\usepackage{graphicx,multicol}
\usepackage{comment,enumerate,hyperref}
\usepackage{amsmath,amsfonts,amssymb}
\usepackage{wasysym}
\usepackage{tikz}
\usepackage[dutch]{babel}
\usepackage[utf8]{inputenc}
\usepackage{multirow}
\usepackage{eurosym}
\usepackage{listings}
\usepackage[T1]{fontenc}
\usepackage{lmodern}
\usepackage{textcomp}
\usepackage{framed}
\usepackage{wrapfig}
\usepackage{pgf-pie}
\usepackage{pgfplots}
\usepackage{booktabs}
\usepackage{pgfplotstable}
\usepackage{changepage}
\usepackage{pst-plot,pst-func}

%---------- Configuratie ------------------------------------------------------

\usetikzlibrary{arrows,shapes,backgrounds,positioning,shadows}
\usetikzlibrary{pgfplots.statistics}
\newif\ifprivate
\privatetrue


\usetheme{hogent}
\setbeameroption{show notes}

%---------- Commando-definities -----------------------------------------------

\newcommand{\tabitem}{~~\llap{\textbullet}~~}
\renewcommand{\arraystretch}{1.2}

%---------- Info over de presentatie ------------------------------------------

\title[Intro]{Onderzoekstechnieken\\Voorbereiding Bachelorproef}
\author{Jens Buysse \and Wim {De Bruyn} \and Wim Goedertier \and Bert {Van Vreckem}}
\date{AJ 2017-2018}

%==============================================================================
% Inhoud presentatie
%==============================================================================

\begin{document}

%---------- Front matter ------------------------------------------------------

% Dia met het HoGent logo
\HoGentLogo

% Titeldia met faculteitslogo
\titleframe

%---------- Inhoud ------------------------------------------------------------

\begin{frame}
  \frametitle{What's on the menu today?}

  \tableofcontents
\end{frame}

\begin{frame}
  \frametitle{Meer info}
  
  \begin{itemize}
    \item \textbf{Algemeen:} Elektronische cursus Onderzoeksvaardigheden (HoGent Bib)
    
    \url{https://bib.hogent.be/how-to/kritisch-denken-en-onderzoekscompetenties/}
    
    \item \textbf{Specifiek voor TIN}: De bachelorproef informatica: Een praktische gids:
    
    \url{https://github.com/HoGentTIN/bachproef-gids/releases}
  \end{itemize}
\end{frame}

\section{De bachelorproef informatica}
\sectionframe{}

\begin{frame}{De bachelorproef informatica}
  \framesubtitle{Doelstelling}
  
  \brightbox{Kunnen advies geven over toepassen nieuwe ict-technologieën in bedrijfscontext}
  
  \begin{itemize}
    \item informatie verzamelen en kritisch beschouwen
    \begin{itemize}
      \item literatuur
      \item belanghebbenden
      \item experimenten
    \end{itemize}
    \item structureren en analyseren
    \item proof-of-concept opzetten
    \item rapporteren
  \end{itemize}

\end{frame}

\begin{frame}{Bachelorproef vs stage}

  \begin{multicols}{2}
    \textbf{Stage}
    
    \begin{itemize}
      \item Uitvoerend
      \item Resultaat kan afgewerkt product zijn
    \end{itemize}
    
    \columnbreak
    
    \textbf{Bachelorproef}
    
    \begin{itemize}
      \item Onderzoekend
      \item Resultaat is rapport, proof-of-concept, \textbf{geen} product
    \end{itemize}
  \end{multicols}

Stage en bachelorproef mogen over zelfde onderwerp gaan, maar moeten gescheiden zijn en onafhankelijk van elkaar.

\end{frame}

\begin{frame}{De bachelorproef informatica}
  \framesubtitle{Praktisch}
  
  \begin{itemize}
    \item 3e modeltraject
    \item 1e semester: onderwerp kiezen en uitwerken
    \item 2e semester: realisatie
    \item 4 dagen stage; 1 dag BP
    \item indienen bij aanvang examenperiode
    \item presentatie + verdediging in juni
  \end{itemize}

\end{frame}

\begin{frame}{De bachelorproef informatica}
  \framesubtitle{Opvolging}
  
  \begin{itemize}
    \item \textbf{Promotor:} \emph{proces}opvolging
    \begin{itemize}
      \item Lector van de opleiding
      \item Voorzitter jury presentatie
      \item Kent examencijfer toe
    \end{itemize}
    \item \textbf{Co-promotor:} \emph{inhoudelijke} opvolging
    \begin{itemize}
      \item Opdrachtgever
      \item Vakexpert
      \item Van buiten HoGent, (evt. een lector)
      \item Geen familielid (tot 3e graad)
    \end{itemize}
  \end{itemize}

Je dient zelf een co-promotor te zoeken!

\end{frame}

\section{Een onderwerp kiezen}
\sectionframe{}

\begin{frame}{Geschikte onderwerpen}

  Bachelorproef = \emph{toegepast} onderzoek
  
  \begin{itemize}
    \item Concreet probleem in bedrijfscontext
    \item Wat is de beste oplossing binnen huidige state-of-the-art?
  \end{itemize}

  \brightbox{Start vanuit een concrete, \textcolor{HoGentAccent6}{reële bedrijfscasus}}
\end{frame}

\begin{frame}{ONgeschikte onderwerpen}

\begin{itemize}
  \item Onvoldoende technische diepgang (bv. enquête)
  \item De toekomst van \ldots (niet speculeren!)
  \item Algemene vergelijking van frameworks/producten/\ldots
  \begin{itemize}
    \item Afhankelijk van specifieke requirements
    \item Koppel dit aan een reële bedrijfscasus
  \end{itemize}
  \item Te breed, vaag of vrijblijvend
  \item Enkel literatuurstudie
  \item Geen eigen bijdrage
\end{itemize}

\end{frame}

\begin{frame}{Op zoek naar een onderwerp}

  \begin{itemize}
    \item Via stagebedrijf/-mentor
    \item (Beperkt) aanbod onderwerpen via Chamilo
    \item Voorstel vanuit je eigen contacten in het werkveld
    \item Eigen idee/voorstel
  \end{itemize}
  
  \brightbox{Wacht niet te lang met het zoeken naar een onderwerp!}
\end{frame}

\begin{frame}{Zelf een onderwerp zoeken}
  \framesubtitle{Kies een onderzoeksdomein}

  \begin{itemize}
    \item Keuzevak 3TI
    \item In welke job wil je starten?
    \item Met welke technologieën/platformen/\ldots wil je werken?
  \end{itemize}

\end{frame}

\begin{frame}{Zelf een onderwerp zoeken}
  \framesubtitle{Volg de actualiteit}

  \begin{itemize}
    \item Portaalsites (dzone, infoq, technet, \ldots)
    \item Relevante conferenties/meetups
    \item Lokale vakverenigingen (bv. OWASP Belgium)
    \item Belangrijkste namen ``community'' volgen
    
    (bv. via blog, Twitter)
    
    \item Newsletters
    \item \ldots
  \end{itemize}

  \brightbox{Start hier nu al mee!}
\end{frame}

\begin{frame}{Een onderzoeksvraag formuleren}
  
  \begin{itemize}
    \item Wat zijn de actuele thema's in de ``community''?
    \begin{itemize}
      \item Onderwerpen op conferenties
      \item Discussies op fora, Twitter, \ldots
    \end{itemize}
    \item Op een onderzoeksvraag is nu nog geen antwoord!
    \item Zoek een co-promotor, vraag raad
  \end{itemize}
  
\end{frame}

\begin{frame}{Vaak gestelde vraag}
  
  Is \ldots een goed onderwerp voor de bachelorproef?
  
  \begin{itemize}
    \item De cloud
    \item Blockchain
    \item AI
    \item Security
    \item \ldots
  \end{itemize}

  \brightbox{Ja, vooropgesteld dat het onderwerp \textcolor{HoGentAccent6}{ICT-gerelateerd is}, \textcolor{HoGentAccent6}{technische diepgang heeft} en je een \textcolor{HoGentAccent6}{concrete onderzoeksvraag} hebt geformuleerd vanuit een \textcolor{HoGentAccent6}{reële bedrijfscasus}}
\end{frame}

\section{Het onderwerp uitschrijven}
\sectionframe{}

\begin{frame}{Het onderwerp uitschrijven}

  \begin{itemize}
    \item Aan de hand van sjabloon
    
    \url{https://github.com/HoGentTIN/bachproef-latex-sjabloon/tree/master/voorstel}
    
    \item Doel: zekerheid scheppen dat je voorstel \emph{S.M.A.R.T.} is
    \begin{itemize}
      \item Specifiek, concreet
      \item Meetbare doelstellingen
      \item Acceptabel voor doelgroep
      \item Realistisch en haalbaar
      \item Tijdgebonden
    \end{itemize}
  \end{itemize}
\end{frame}

\begin{frame}{Procedure}

\begin{itemize}
  \item Indienen op Chamilo (cursus Bachelorproef) vóór deadline (half december)
  \item Feedback door één van de lectoren, dit wordt je promotor
  \item Herwerk zo nodig voorstel
  \item Aan de slag!
\end{itemize}
\end{frame}

\section{Criteria voor beoordeling voorstel}
\sectionframe{}

\begin{frame}{Criteria voor beoordeling voorstel}

De \textbf{titel} is \emph{concreet} en geeft een correct beeld van de onderzoeksdoelstelling

\begin{itemize}
  \item Niet enkel het onderzoeksdomein benoemen
  \item Geen vraag
  \item Geen ``Onderzoek naar''
\end{itemize}

% TODO: voorbeelden geven: goed/slecht

%\textcolor{OrangeRed}{\XBox} De cloud
%
%\textcolor{ForestGreen}{\CheckedBox} Selectie van een Platform-as-a-Service oplossing voor een 

\end{frame}

\begin{frame}{Criteria voor beoordeling voorstel}

Er is een \textbf{samenvatting} met alle gevraagde elementen:

\begin{description}
  \item[Context] Waarom is dit werk belangrijk?
  \item[Nood] Waarom moet dit onderzocht worden?
  \item[Taak] Wat ga je specifiek doen doen?
  \item[Object] Wat staat in dit document geschreven?
  \item[Resultaat] Welk concreet resultaat verwacht je van je onderzoek?
  \item[Conclusie] Wat verwacht je van van de conclusies?
  \item[Perspectief] Waarvoor zal het resultaat kunnen gebruikt worden? Wat is de meerwaarde van je werk?
\end{description}

\brightbox{Een samenvatting is geen inleiding!}

\end{frame}

\begin{frame}{Criteria voor beoordeling voorstel}

Voorstel is \textbf{vernieuwend} en heeft een duidelijke \textbf{meerwaarde} voor een specifieke doelgroep uit het werkveld

\begin{itemize}
  \item Op een onderzoeksvraag is nu nog geen antwoord!
  \item Eigen bijdrage moet duidelijk zijn!
  \item Baken domein af naar specifieke doelgroep toe (bv. een bedrijf)
  \item Geen ``algemeen'' onderwerp
\end{itemize}

\end{frame}

\begin{frame}{Criteria voor beoordeling voorstel}

De methodologie is duidelijk verantwoord, onderzoekstechnieken zijn geschikt voor beantwoorden onderzoeksvraag

\begin{itemize}
  \item Concreet plan van aanpak!
  \item Uit welke fasen bestaat jouw onderzoek?
  \item Welk doel en resultaat heeft elke fase?
  \item Wees specifiek! Niet: ``onderzoek doen naar\ldots''
  \begin{itemize}
    \item Literatuurstudie
    \item Interview (bv. requirements verzamelen)
    \item Experiment (bv. performantiemeting)
    \item Vergelijkende studie
    \item Risico-analyse (bv. in kader van security)
    \item Proof-of-concept opzetten
    \item \ldots
  \end{itemize}
\end{itemize}

\end{frame}

\begin{frame}{Criteria voor beoordeling voorstel}

Er wordt op een correcte manier verwezen naar vakliteratuur.

\begin{itemize}
  \item Voldoende referenties
  \item Elke bewering
  \item APA-stijl (volgens sjabloon)
\end{itemize}

\end{frame}

\begin{frame}{Criteria voor beoordeling voorstel}

Er wordt een \textbf{zakelijke schrijfstijl} gehanteerd

\begin{description}
  \item[Zakelijk] geen bloemrijk, informeel taalgebruik, spreektaal
  \item[Objectief] geen mening, enkel aantoonbare feiten
  \item[Onpersoonlijk] \textbf{Geen ``ik''}/wij
  \item[Precies] geen vage uitspraken, alles kwantificeren
  \item[Correct] taalgebruik: grammatica, spelling
\end{description}

Zie \url{http://www.taalwinkel.nl/een-wetenschappelijke-schrijfstijl/}

\end{frame}

\section{Veel succes!}
\sectionframe{}

\end{document}

