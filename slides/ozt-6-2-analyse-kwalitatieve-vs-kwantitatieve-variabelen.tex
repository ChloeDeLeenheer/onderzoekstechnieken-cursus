%%----------------------------------------------------------------------------
%% Onderzoekstechnieken: Analyse van kwalitatieve vs kwantitatieve variabele
%%----------------------------------------------------------------------------

\documentclass[aspectratio=169]{beamer}

%==============================================================================
% Aanloop
%==============================================================================

%---------- Vormgeving --------------------------------------------------------

\usetheme{hogent}

\usecolortheme{hgwhite} % witte achtergrond, zwarte tekst

\usepackage{graphicx,multicol}
\usepackage{comment,enumerate,hyperref}
\usepackage{amsmath,amsfonts,amssymb}
\usepackage[dutch]{babel}
\usepackage{multirow}
\usepackage{eurosym}
\usepackage{listings}
\usepackage{textcomp}
\usepackage{framed}
\usepackage{wrapfig}
\usepackage{tabu} %needed for \tabulinesep
\usepackage{wrapfig}
\usepackage{pgf-pie}
\usepackage{pgfplots}
\usepackage{booktabs}
\usepackage{pgfplotstable}
\usepackage{changepage}
\usepackage{ulem} % for \sout{text} (strikethrough)
\usepackage{fancyvrb} % for \begin{Verbatim} (LaTeX controls within verbatim)

%---------- Configuratie ------------------------------------------------------

\pgfplotsset{compat=1.16}
\usetikzlibrary{arrows,shapes,backgrounds,positioning,shadows}
\usetikzlibrary{pgfplots.statistics}

%---------- Commando-definities -----------------------------------------------

\newcommand{\tabitem}{~~\llap{\textbullet}~~}
\newcommand{\alertbox}[2][hgblue]{%
  \setbeamercolor{alertbox}{bg=#1,fg=white}
  \begin{beamercolorbox}[sep=2pt,center]{alertbox}
    \textbf{#2}
  \end{beamercolorbox}
}
\pgfmathdeclarefunction{gauss}{2}{%
  \pgfmathparse{1/(#2*sqrt(2*pi))*exp(-((x-#1)^2)/(2*#2^2))}%
}

%---------- Info over de presentatie ------------------------------------------

\title{Hst 6. Analyse van kwalitatieve vs kwantitatieve variabelen}
\subtitle{Onderzoekstechnieken}
\author{Jens Buysse \and Wim {De Bruyn} \and Bert {Van Vreckem} \and Pieterjan Maenhaut}
\date{AJ 2020-2021}

%==============================================================================
% Inhoud presentatie
%==============================================================================

\begin{document}

\begin{frame}
  \maketitle
\end{frame}

\begin{frame}
  \frametitle{What's on the menu today?}
  
  \tableofcontents
\end{frame}

\begin{frame}
  \frametitle{Leerdoelen}
  
  \begin{itemize}
    \item $t$-test voor twee steekproeven kunnen toepassen
    \item Effectgrootte kunnen berekenen ahv Cohen's $d$
  \end{itemize}
\end{frame}

\begin{frame}
  \frametitle{Overzicht}
  \centering
  \begin{tabular}{lll}
    \toprule
    \textbf{Onafhankelijke} & \textbf{Afhankelijke} & \textbf{Toets/metriek}        \\
    \midrule
    Kwalitatief             & Kwalitatief           & $\chi^2$-toets                \\
    &                       & Cramér's $V$                  \\
    Kwalitatief             & Kwantitatief          & $t$-toets voor 2 steekproeven \\
    &                       & Cohen's $d$                   \\
    Kwantitatief            & Kwantitatief          & ---                           \\
    &                       & Regressie, correlatie         \\
    \bottomrule
  \end{tabular}
\end{frame}

\section{De t-toets voor 2 steekproeven}

\begin{frame}
  \frametitle{Vergelijken van twee steekproeven}
  
  Voorbeelden onderzoeksvragen:
  
  \begin{itemize}
    \item Verdienen mannen meer dan vrouwen?
    \item Kan je beter studeren zonder smartphone in de buurt?
    \item Halen studenten uit het ASO betere resultaten als die uit het TSO?
    \item Werkt een medicijn effectief?
    \item \ldots
  \end{itemize}

  Wat is telkens de onafhankelijke en afhankelijke variabele?
\end{frame}

\begin{frame}
  \frametitle{Vergelijken van twee steekproeven}
  
  Is steekproefgemiddelde van twee steekproeven significant verschillend?
  
  \begin{itemize}
    \item Onafhankelijke steekproeven
    \item Gepaarde steekproeven
  \end{itemize}
\end{frame}

\begin{frame}
  \frametitle{Onafhankelijke steekproeven}
  
  In een klinisch onderzoek wil men nagaan of een nieuw medicijn als bijwerking een verminderde reactiesnelheid heeft.
  
  \begin{itemize}
    \item Controlegroep: 6 deelnemers krijgen placebo
    \item Interventiegroep: 6 deelnemers krijgen medicijn
  \end{itemize}
  
  \pause
  
  Vervolgens wordt reactiesnelheid gemeten:
  
  \begin{itemize}
    \item Controlegroep: 91, 87, 99, 77, 88, 91 ~~~~~~~~~~~($\overline{x}=88,83$)
    \item Interventiegroep: 101, 110, 103, 93, 99, 104 ~~($\overline{y}=101,67$)
  \end{itemize}
  
  Zijn er significante verschillen tussen de interventie- en controlegroep?
\end{frame}

\begin{frame}[fragile]
  \frametitle{Onafhankelijke steekproeven}
  \begin{enumerate}
    \item Hypothese:
    \begin{itemize}
      \item $H_0$: $\mu_1 - \mu_2 = 0$
      \item $H_1$: $\mu_1 - \mu_2 < 0$
    \end{itemize}
    \item Significantieniveau: $\alpha = 0,05$
    \item Toetsingsgrootheid:
    \begin{itemize}
      \item $\overline{x}-\overline{y} = -12,833$
      \item $\overline{x} =$ schatting voor $\mu_1$ (controlegroep) 
      \item $\overline{y} =$ schatting voor $\mu_2$ (interventiegroep) 
    \end{itemize}
  \end{enumerate}
  \vfill
  In R:
  {\footnotesize
    \begin{Verbatim}[commandchars=\\\{\}]
    > controle <-  c(91, 87, 99, 77, 88, 91)
    > interventie <- c(101, 110, 103, 93, 99, 104)
    > t.test(controle, interventie, alternative="less",
    mu=0, conf.level=0.95)
    \end{Verbatim}
  }
\end{frame}

\begin{frame}[fragile]
  \frametitle{Onafhankelijke steekproeven}
  
  {\footnotesize
    \begin{Verbatim}[commandchars=\\\{\}]
    Welch Two Sample t-test
    
    data:  controle and interventie
    t = -3.4456, df = 9.4797, p-value = 0.003391
    alternative hypothesis: true difference in means is less than 0
    \sout{95 percent confidence interval:}      {\tiny \textcolor{red}{OPGELET!} Dit heeft niets te maken met}
    \sout{-Inf -6.044949}                       {\tiny   aanvaardingsgebied of kritiek gebied}
    sample estimates:
    mean of x mean of y 
    88.83333 101.66667
    \end{Verbatim}
  }
  \vfill
  $\overline{x}-\overline{y}=-12,833$ komt overeen met $t=-3,4456$\\
  $df=9,48$ wordt berekend door \texttt{t.test()} op basis van $x$ en $y$\\
  $p = 0,003391 < \alpha = 0,05$ dus $H_0$ verworpen (significant verschil)
\end{frame}

\begin{frame}
  \frametitle{Gepaarde steekproef}
  
  In een studie werd nagegaan of auto's die rijden op benzine met additieven ook een lager verbruik hebben.
  
  Bij 10 auto's werd het verbruik gemeten (uitgedrukt in mijl per gallon) voor beide soorten benzine:
  
  \vspace{.5cm}
  \centering
  \begin{tabular}{|l|c|c|c|c|c|c|c|c|c|c|}
    \hline
    Auto           & 1  & 2  & 3  & 4  & 5  & 6  & 7  & 8  & 9  & 10 \\ \hline
    Gewone benzine & 16 & 20 & 21 & 22 & 23 & 22 & 27 & 25 & 27 & 28 \\ \hline
    Met additieven & 19 & 22 & 24 & 24 & 25 & 25 & 25 & 26 & 28 & 32 \\ \hline
  \end{tabular} 
\end{frame}

\begin{frame}[fragile]
  \frametitle{Gepaarde steekproef}
  \begin{enumerate}
    \item Hypothese:
    \begin{itemize}
      \item $H_0$: $\overline{x-y} = 0$
      \item $H_1$: $\overline{x-y} > 0$
    \end{itemize}
    \item Significantieniveau: $\alpha = 0,05$
    \item Toetsingsgrootheid:
    \begin{itemize}
      \item $\overline{x-y}$
      \item $x =$ mijl per gallon met additieven ($\overline{x}=25,1$)
      \item $y =$ mijl per gallon met gewone benzine ($\overline{y}=23,1$)
    \end{itemize}
  \end{enumerate}
  \vfill
  In R:
  {\footnotesize
    \begin{Verbatim}[commandchars=\\\{\}]
    > gewone    <- c(16, 20, 21, 22, 23, 22, 27, 25, 27, 28)
    > additieven <-c(19, 22, 24, 24, 25, 25, 26, 26, 28, 32)
    > t.test(additieven, gewone, alternative="greater",
    \textcolor{red}{paired=TRUE}, mu=0, conf.level=0.95)
    \end{Verbatim}
  }
\end{frame}

\begin{frame}[fragile]
  \frametitle{Gepaarde steekproef}
  
  {\footnotesize
    \begin{Verbatim}[commandchars=\\\{\}]
    Paired t-test
    
    data:  additieven and gewone
    t = 4.4721, df = 9, p-value = 0.0007749
    alternative hypothesis: true difference in means is greater than 0
    \sout{95 percent confidence interval:}      {\tiny \textcolor{red}{OPGELET!} Dit heeft niets te maken met}
    \sout{1.180207      Inf}                    {\tiny   aanvaardingsgebied of kritiek gebied}
    sample estimates:
    mean of the differences 
    2
    \end{Verbatim}
  }
  \vfill
  $\overline{x}-\overline{y}=2$ komt overeen met $t=4,4721$\\
  $p = 0,0007749 < \alpha = 0,05$ dus $H_0$ verworpen (significant verschil)
\end{frame}

\section{Effectgrootte}

\begin{frame}
  \frametitle{Effectgrootte}
  
  \alertbox{De \textcolor{hgyellow}{effectgrootte} is een metriek die uitdrukt hoe groot het verschil tussen twee groepen is}
  
  \begin{itemize}
    \item Controlegroep vs. interventiegroep
    \item Kan je gebruiken naast hypothesetoets
    \item Vaak gebruikt in onderwijswetenschappen
    \item Er zijn verschillende definities, hier: \textit{Cohen's $d$}
  \end{itemize}
\end{frame}

\begin{frame}
  \frametitle{Cohen's $d$}
  
  \[ d = \frac{\overline{x_1} - \overline{x_2}}{s} \]
  
  met $\overline{x_1}$, $\overline{x_2}$ gemiddelde van de steekproeven
  
  en $s$ een gecombineerde standaardafwijking over beide groepen:
  
  \[ s = \sqrt{\frac{(n_1 - 1) s_1^2 + (n_2 - 1) s_2^2}{n_1 + n_2 -2}} \]
  
  met $n_1$, $n_2$ de steekproefgroottes, $s_1$, $s_2$ de standaardafwijking van beide groepen
  
\end{frame}

\begin{frame}
  \frametitle{Interpretatie Cohen's $d$}
  
  \begin{columns}
    \column{.5\textwidth}
    
    \begin{center}
      \begin{tabular}{ll}
        \toprule
        $|d|$  & \textbf{Effectgrootte} \\
        \midrule
        0,01 & zeer klein             \\
        0,2  & klein                  \\
        0,5  & matig                  \\
        0,8  & groot                  \\
        1,2  & zeer groot             \\
        2,0  & enorm                  \\
        \bottomrule
      \end{tabular}
    \end{center}
    
    \column{.5\textwidth}
    
    In onderwijswetenschappen (John Hattie):
    
    \begin{itemize}
      \item 0,4 = kantelpunt voor gewenste effecten
      \item effectgrootte $d = 1$: leerstof van 1j op 6m verwerken!
    \end{itemize}
    
    Zie bvb. \url{http://www.evidencebasedteaching.org.au/hatties-2017-updated-list/}
  \end{columns}
\end{frame}

\begin{frame}
  \frametitle{Typisch opzet onderzoek in onderwijs}
  
  \begin{itemize}
    \item Onderzoeksvraag: Is X een goede leerstrategie, m.a.w. heeft dit een positief effect op eindresultaten?
    \item Controlegroep gebruikt ``normale'', ``klassieke'' techniek
    \item In de interventiegroep wordt X toegepast
    \item Achteraf volgt evaluatiemoment
    \item Scores bepalen, $d$ berekenen
  \end{itemize}
\end{frame}

\end{document}