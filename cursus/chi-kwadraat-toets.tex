\chapter{De \texorpdfstring{$\chi^{2}$}{Chi-kwadraat} toets}
\label{ch:chikwadraat}

\section{\texorpdfstring{$\chi^{2}$}{Chi-kwadraat} toets voor verdelingen}


\section{Oefeningen}
\label{sec:chi-kwadraat-oefeningen}

\begin{exercise}
  \label{ex:chisq-survey}
  Voor deze oefening maken we gebruik van de dataset \texttt{survey} die is meegeleverd met R. De dataset is samengesteld uit een bevraging onder studenten. Om deze te laden, doe het volgende:
  
  \begin{lstlisting}
  library(MASS)
  View(survey)  # Toont de "survey" dataset
  ?survey       # Help-pagina voor deze dataset met uitleg over de inhoud
  \end{lstlisting}
  
  Als je een foutboodschap krijgt bij het laden van de bibliotheek (eerste regel), betekent dit dat de package \texttt{MASS} nog niet geïnstalleerd is. Dit kan je alsnog doen via Tools > Install Packages en het invullen van de package-naam in het tekstveld.
  
  We willen de relatie onderzoeken tussen enkele discrete (nominale of ordinale) variabelen in deze dataset. Voor elke hieronder opgesomde paren, volg deze stappen:
  
  \begin{enumerate}[label=(\alph*)]
    \item Denk eerst eens na welke uitkomt je precies verwacht voor de opgegeven combinatie van variabelen.
    \item Stel een frequentietabel op voor de twee variabelen. De (vermoedelijk) onafhankelijke variabele komt eerst.
    \item Plot een grafiek van de data, bv.~geclusterde staafgrafiek, gestapelde staafgrafiek van relatieve frequenties, of een ``mozaïekgrafiek'' (eenvoudig met \texttt{plot(table(data\$col1, data\$col2))}).
    \item Als je de grafiek bekijkt, verwacht je dan een eerder hoge of eerder lage waarde voor de $\chi^2$-statistiek? Waarom?
    \item Bereken de $\chi^2$-statistiek en de kritieke grenswaarde $g$ (voor significantieniveau $\alpha = 0.05$)
    \item Bereken de $p$-waarde
    \item Moeten we de nulhypothese aanvaarden of verwerpen? Wat betekent dat concreet voor de relatie tussen de twee variabelen?
  \end{enumerate}

  Hieronder zijn de te onderzoeken variabelen opgesomd. De vermoedelijke onafhankelijke variabele komt telkens eerst.
  
  \begin{enumerate}
    \item \texttt{Exer} (sporten) en \texttt{Smoke} (rookgedrag)
    \item \texttt{W.Hnd} (de hand waarmee je schrijft) en \texttt{Fold} (de hand die bovenaan komt als je de armen kruist)
    \item \texttt{Sex} (gender) en \texttt{Smoke}
    \item \texttt{Sex} en \texttt{W.Hnd}
  \end{enumerate}
\end{exercise}

\begin{exercise}
  \label{ex:chisq-aids2}
  Laad de dataset \texttt{Aids2} uit package \texttt{MASS} (zie Oefening~\ref{ex:chisq-survey}) die informatie bevat over 2843 patiënten die vóór 1991 in Australië met AIDS besmet werden. Deze dataset werd in detail besproken door~\textcite{Ripley2007}. Onderzoek of er een relatie is tussen de variabele geslacht (\texttt{Sex}) en de manier van besmetting (\texttt{T.categ}).
  
  \begin{enumerate}
    \item Ga op de gebruikelijke manier te werk: visualiseren van de data, $\chi^2$, $g$ en $p$-waarde berekenen ($\alpha = 0,05$), en tenslotte een conclusie formuleren.
    \item Bepaal de gestandaardiseerde residuën om te bepalen welke categorieën extreme waarden bevatten.
  \end{enumerate}
  
\end{exercise}

\begin{exercise}
  \label{ex:chisq-digimeter}
  
  Elk jaar voert Imec (voorheen iMinds) een studie uit over het gebruik van digitale technologieën in Vlaanderen, de Digimeter~\autocite{Vanhaelewyn2016}. In deze oefening zullen we nagaan of de steekproef van de Digimeter 2016 ($n = 2164$) representatief is voor de bevolking wat betreft de leeftijdscategorieën van de deelnemers.
  
  In Tabel~\ref{tab:digimeter2016} worden de relatieve frequencies van de deelnemers weergegeven. De absolute frequenties voor de verschillende leeftijdscategorieën van de Vlaamse bevolking worden samengevat in Tabel~\ref{tab:leeftijd-vlaanderen}. Deze gegevens zijn ook te vinden in bijgevoegd CSV-bestand \texttt{oefeningen/data/bestat-vl-ages.csv}.
  
  \begin{enumerate}
    \item De tabel met leeftijdsgegevens van de Vlaamse bevolking als geheel heeft meer categorieën dan deze gebruikt in de Digimeter. Maak een samenvatting zodat je dezelfde categorieën overhoudt dan deze van de Digimeter. Tip: dit gaat misschien makkelijker in een rekenblad dan in R.
    \item Om de goodness-of-fit test te kunnen toepassen hebben we de absolute frequenties nodig van de geobserveerde waarden in de steekproef. Bereken deze.
    \item Bereken ook de verwachte percentages ($\pi_{i}$) voor de populatie als geheel.
    \item Voer de goodness-of-fit test uit over de verdeling van leeftijdscategorieën in de steekproef van de Digimeter. Is de steekproef in dit opzicht inderdaad representatief voor de Vlaamse bevolking?
  \end{enumerate}
\end{exercise}

\begin{table}
  \caption{Frequenties van de leeftijd van deelnemers aan de iMec Digimeter 2016 en de Vlaamse bevolking.}
  \label{tab:frequenties-leeftijden}
  \centering
  \begin{tabular}{cc}
    \textbf{Leeftijdsgroep} & \textbf{Percentage} \\ \midrule
    15-19 & 6,6\% \\
    20-29 & 14,2\% \\
    30-39 & 15,0\% \\
    40-49 & 16,3\% \\
    50-59 & 17,3\% \\
    60-64 & 7,3\% \\
    64+   & 23,2\% \\
  \end{tabular}
  \subcaption{Percentage van deelnemers aan de Digimeter 2016 van iMec ($n = 2164$), opgedeeld per leeftijdscategorie. \autocite{Vanhaelewyn2016}}
  \label{tab:digimeter2016}
  
  \centering
  \begin{tabular}{cc}
    \textbf{Leeftijdsgroep} & \textbf{Aantal} \\ \midrule
              –5            &     352017      \\
              5-9           &     330320      \\
             10-14          &     341303      \\
             15-19          &     366648      \\
             20-24          &     375469      \\
             25-29          &     387131      \\
             30-34          &     401285      \\
             35-39          &     409587      \\
             40-44          &     458485      \\
             45-49          &     493720      \\
             50-54          &     463668      \\
             55-59          &     413315      \\
             60-64          &     379301      \\
             65-69          &     299152      \\
             70-74          &     279789      \\
             75-79          &     249260      \\
             80-84          &     182352      \\
             85-89          &     104449      \\
             90-94          &      29888      \\
             95-99          &      7678       \\
             100+           &       923
  \end{tabular}
  \subcaption{Absolute frequentie van de Vlaamse bevolking per leeftijdscategorie. Bron: BelStat (\url{https://bestat.economie.fgov.be/bestat/}, C01.1: Bevolking volgens verblijfplaats (provincie), geslacht, positie in het huishouden (C), burgerlijke staat en leeftijd (B)).}
  \label{tab:leeftijd-vlaanderen}
  

\end{table}

\section{Antwoorden op geselecteerde oefeningen}
\label{sec:chi-kwadraat-oplossingen}

\paragraph{Oefening~\ref{ex:chisq-survey}}

\begin{enumerate}
  \item \texttt{Exer}/\texttt{Smoke}: $\chi^2 = 5.4885$, $g = 12.59159$, $p = 0.4828422$
  \item \texttt{W.Hnd}/\texttt{Fold}: $\chi^2 = 1.581399$, $g = 5.9915$, $p = 0.454$
  \item \texttt{Sex}/\texttt{Smoke}: $\chi^2 = 3.554$, $g = 7.8147$, $p = 0.314$
  \item \texttt{Sex}/\texttt{W.Hnd}: $\chi^2 = 0.236$, $g = 3.8415$, $p = 0.627$
\end{enumerate}

\paragraph{Oefening~\ref{ex:chisq-aids2}} $\chi^2 = 1083.372914$, $g = 14.067140$, $p \approx 1.157 \times 10^{-229}$

\paragraph{Oefening~\ref{ex:chisq-digimeter}} $\chi^2 \approx 6.6997$ ($df = 6$), $g \approx 12.5916$, $p \approx 0.3495$
