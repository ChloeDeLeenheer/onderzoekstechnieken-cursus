\chapter{Notatie}

\begin{table}
  \centering
  \begin{tabular}{p{.25\textwidth}p{.75\textwidth}}
    \toprule
    \textbf{Notatie} & \textbf{Betekenis} \\
    \midrule
    $X$                       & Een stochastische variabele \\
    $N$                       & De populatieomvang \\
    $n$                       & De steekproefgrootte \\
    $\mu$ (mu)                & Het gemiddelde (ook: verwachtingswaarde) over heel de \emph{populatie}. \\
    $\overline{x}$            & Het gemiddelde over de \emph{steekproef} \\
    $\sigma$ (sigma)          & De standaardafwijking over heel de populatie \\
    $s$                       & De standaardafwijking over de steekproef \\
    $X \sim Nor(\mu, \sigma)$ & De variabele $X$ is \emph{normaal verdeeld} met gemiddelde $\mu$ en standaardafwijking $\sigma$ \\
    $Z \sim Nor(0, 1)$        & $Z$ is een variabele met een kansverdeling die de \emph{standaardnormaalverdeling} volgt, dus met gemiddelde 0 en standaardafwijking 1 \\
    $M$                       & De kansverdeling van het populatiegemiddelde, op basis van een steekproef (cfr.~de centrale limietstelling, Sectie~\ref{sec:centrale-limietstelling}) \\

    \bottomrule
  \end{tabular}
\end{table}
