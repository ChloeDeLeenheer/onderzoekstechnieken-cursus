\chapter{Analyse op 1 variabele}
\label{ch:analyse1var}

%% TODO: volgorde van centrum-/ spreidingsmaten eens bekijken en overzichtelijker maken.

\begin{definition}[Beschrijvende statistiek]
  Met beschrijvende statistiek \index{beschrijvende statistiek} bedoelen we een verzameling van technieken om data synthetisch voor te stellen en samen te vatten.
\end{definition}

\section{Voorbeeld met superhelden}
\begin{table}
  \centering
  \begin{tabular}{|c|c|c|c|c|}
    \hline
    $x_{1}$ & $x_{2}$ & $x_{3}$ & $x_{4}$&  $x_{5}$ \\
    \hline
    141 & 198 & 143 & 201 & 184 \\
    \hline
  \end{tabular}
  \caption{Voorbeeldtabel superhelden vanuit slides}
  \label{tab:helden}
\end{table}

\section{Gemiddelde}
\begin{definition}[Gemiddelde]
  \index{Gemiddelde} Het gemiddelde (symbool $\mu$) van een set waarden is de som van al deze waarden gedeeld door het aantal waarden. De formule staat beschreven in \ref{eq:Mean}.
  \begin{equation}
    \mu = \frac{1}{n} \times \sum_{i=1}^{n} x_{i}
    \label{eq:Mean}
  \end{equation}

  Waarbij:
  \begin{itemize}
    \item $x_{i}$ de waarden zijn vanuit tabel \ref{tab:helden}.
    \item $n$ het aantal waarden is. In het voorbeeld van de superhelden zou dit 5 zijn, want we hebben 5 lengtes van superhelden.
  \end{itemize}
\end{definition}


\begin{exercise}
  Wat is de gemiddelde lengte van de superhelden?
\end{exercise}

\begin{exercise}
  Vraag: het gemiddelde van 15 cijfers is 12. Welk nummer moeten
  we aan de rij van cijfers toevoegen om een gemiddelde van 13 te bekomen?
\end{exercise}

Het rekenkundig gemiddelde is gevoelig aan outliers: een extreme waarde kan het rekenkundig gemiddelde zwaar be\"invloeden.

\section{Mediaan}

\begin{definition}[Mediaan]
  Indien we alle cijfers sorteren van klein naar groot, is de \index{Mediaan} mediaan het middelste cijfer, of het gemiddelde van de twee middelste cijfers indien het aantal cijfers oneven is.
\end{definition}

De mediaan is niet gevoelig aan outliers.

\section{Modus}
\begin{definition}[Modus]
  De \index{Modus} modus is het cijfer dat het meest voorkomt in een set van cijfers.
\end{definition}

\begin{itemize}
  \item Heeft niet veel zin als alle cijfers even veel voorkomen. (Zoals bij onze superhelden). Misschien is het nuttig om ze dan te groeperen.
  \item Er kunnen twee modi zijn: dit noemen we \index{Bimodaal} bimodaal;
  \item Er kunnen meerdere modi zijn: dit noemen we \index{Multimodaal} multimodaal.
\end{itemize}

\begin{example}
  Het groeperen kunnen we tonen bijvoorbeeld bij het aantal mensen gered door Batman de laatste acht jaar.
  \begin{itemize}
    \item $[0-9]$ mensen : 4, 7
    \item $[10-19]$ mensen: 11, 16
    \item $[20-29]$ mensen : 20, 22, 25, 26
    \item $[30-39]$ mensen: 33
  \end{itemize}
  Dus categorie $[20-29]$ komt het meest voor. We kunnen dus bv. kiezen om 25 als modus te gebruiken. Zo'n klasse noemen we dan een modale klasse.
\end{example}

\section{Range / Bereik}
\begin{definition}[Bereik]
  Het \index{Range} \index{Bereik} bereik in een set van getallen is de absolute waarde van het verschil tussen het laagste en grootste getal.
\end{definition}

\section{Kwartielen \& kwartielafstand}
\begin{definition}[Kwartielen \& Kwartielafstand]
  De \index{Kwartiel} kwartielen zijn de waarden die een gesorteerde lijst van nummers in 4 gelijke delen deelt. Elk deel vormt dus een kwart van de dataset. Men spreekt van een eerste, tweede en derde kwartiel ($Q_{1}$, $Q_{2}$, $Q_{3}$).
\end{definition}

Dus:
\begin{itemize}
  \item eerste kwartiel $Q_{1}$ is de getalswaarde die de laagste 25 \% van de reeks afscheidt.
  \item tweede kwartiel $Q_{2}$ is de getalwaarde die de laagste 50\% van de reeks afscheidt.
  \item derde kwartiel $Q_{3}$ is de getalwaarde die de laagste 75\% van de reeks afscheidt.
\end{itemize}

\begin{definition}
  \index{Kwartielafstand} Kwartielafstand is het verschil tussen $Q_{3}$ en $Q_{1}$ ( dus $Q_{3} - Q_{1}$).
\end{definition}

Methode om te berekenen (volgens \textcite{Moore2002}) (met $n$ oneven aantal getallen):
\begin{itemize}
  \item $Q_{1}$ komt overeen met cijfer $\frac{n+1}{4}$
  \item $Q_{3}$ komt overeen met cijfer $\frac{3n+3}{4}$
\end{itemize}

Methode om te berekenen (met $n$ even aantal getallen):
\begin{itemize}
  \item $Q_{1}$ komt overeen met cijfer $\frac{n+2}{4}$
  \item $Q_{3}$ komt overeen met cijfer $\frac{3n+2}{4}$
\end{itemize}

\begin{exercise}
  Met welke voorgaande statistiek komt $Q_{2}$ overeen?
\end{exercise}

\section{Variantie en standaardafwijking}
\label{sec:varEnSD}
\begin{definition}[Variantie]
  De \index{Variantie} variantie (symbool $\sigma^{2}$ - lees sigma kwadraat) is het gemiddelde van de kwadraten van de verschillen tussen de waarde van de dataset en het gemiddelde.
  \begin{equation}
    \sigma^{2} = \frac{1}{n} \times \sum_{i=1}^{n} \left( \mu - x_i \right)^{2}
    \label{eq:variantie}
  \end{equation}
\end{definition}


\begin{example}
  De variantie bij de lengtes van onze superhelden wordt als volgt berekend:

  \begin{equation}
    \begin{aligned}
      \sigma^{2} &=  \frac{(173.4-141)^{2} + (173.4 - 198 )^{2} + (173.4 - 143)^{2} + (173.4- 201)^{2} + (173.4  -184 )^{2}}{5} \\
      &=  \frac{(-32.4)^{2}+	(24.6)^{2}	+ (-30.4)^{2}+	(27.6)^{2}	+ (10.6)^{2}}{5}\\
      &= \frac{1049.76 + 	605.16	 + 924.16	 + 761.76 + 	112.36}{5}\\
      &= \frac{3453.2}{5} = 690.64
    \end{aligned}
  \end{equation}
\end{example}

\begin{definition}[Standaardafwijking]
  De \index{Standaardafwijking} standaardafwijking wordt dan gedefinieerd als de vierkantswortel van de variantie.
  \begin{equation}
    \sigma = \sqrt{\sigma^{2}}
    \label{eq:stdev}
  \end{equation}
\end{definition}


Dit geeft ons dus inzicht in wat normaal is en wat abnormaal is: een kleine standaardafwijking wijst erop dat de waarden dicht bij de centrummaat ($\mu$) liggen, terwijl een grote standaardafwijking duidt dat de waarden verspreid liggen over een groot bereik van waarden. In sommige gevallen wil men een grote standaardafwijking, in andere gevallen niet zoals hieronder beschreven.

\begin{example}
  Bij het vervaardigen van een schroevendraaier is de grootte van de kop belangrijk voor het goed functioneren van de schroevendraaier. Als we dus van 100 verschillende schroevendraaiers de kopgrootte meten, is het beter dat die grootte redelijk constant is en wensen we dus een kleine standaardafwijking.
\end{example}

\begin{example}
  Bij het onderzoek naar onze superhelden, wensen we te weten hoeveel ze ongeveer verdienen in hun normale job. We hebben een aantal rijke superhelden (bv. Batman) en een aantal minder rijke superhelden (bv. Spiderman). De spreiding op hun inkomen is dus groot, maar dat is niet per definitie slecht.
\end{example}


Een aangename eigenschap van de standaardafwijking is dat het uitgedrukt kan worden in dezelfde metriek als de gemeten data. Bij ons voorbeeld van de superhelden, wil dat zeggen dat de standaardafwijking 26.28 cm is.

\begin{exercise}
  Waarom nemen we het gemiddelde van het kwadraat van het verschil. Waarom bv. niet de gewone getallen of gemiddelde absolute waarde van de afwijking? Probeer dit eens op het voorbeeld in de slides.
\end{exercise}

Antwoord:
\begin{itemize}
  \item Bij de gewone waarde zouden de negatieve de positieve getallen uitcancellen en dus 0 tot resultaat hebben.
  \item Bij absolute waarde komen beide voorbeelden dezelfde spreiding uit, alhoewel voorbeeld twee meer verspreid ligt.
  \item Bij kwadratuur hebben we vorige problemen niet.
\end{itemize}

Zoals het gemiddelde is de variantie en de standaarddeviatie gevoelig aan outliers (uitschieters). De variantie is eigenlijk gevoeliger dan het gemiddelde. Inderdaad, voor een outlier is de afstand tot het gemiddelde kleiner dan het kwadraat van deze afstand.

\section[Centrum- en spreidingsmaten toepassen]{Toepassing spreidingsmaten en maten centraliteit op verschillende soorten variabelen}

\begin{table}[htbp]
  \centering
  \begin{tabular}{|l|l|l|l|l|}
    \hline
    \textbf{Analyse} & \textbf{Nominaal} & \textbf{Ordinaal} & \textbf{Interval} of \textbf{Ratio} \\
    \hline
    \textbf{Centrum} & Modus & Mediaan & Gemiddelde \\
    & Modale klasse & Modus & Mediaan \\
    & & Modale klasse & Modale klasse \\
    \hline
    \textbf{Spreiding} & & Range & Range \\
    & & Interkwartielafstand & Interkwartielafstand \\
    & & & Standaarddeviatie \\
    \hline
  \end{tabular}
  \caption{Meetniveaus en mogelijkheden op variabelen}
  \label{tab:Meetniveaus}
\end{table}

\section{Grafieken}

\subsection{Boxplot}

De \index{Boxplot} boxplot wordt gevormd door een rechthoek begrensd door de kwartielwaarden (25\% en 75\%). In deze rechthoek wordt ook de mediaan getekend. De stelen, die aan de rechthoek zitten, bevatten de rest van de waarnemingen op de uitschieters en extremen na.

\begin{itemize}
  \item Een \index{Uitschieter} uitschieter is een waarde die meer dan 1.5 keer de interkwartielafstand boven/onder het derde/eerste kwartiel ligt. Wordt aangeduid met een cirkeltje.
  \item Een \index{Extremum} extremum is een waarde die meer dan 3 keer de interkwartielafstand boven/onder het derde/eerste kwartiel ligt. Wordt aangeduid met een sterretje.
\end{itemize}

\chapter{Oefeningen op 1 variabele}
\section{Centrummaten en spreidingsmaten op papier}
\begin{definition}
	Een frequentietabel is tabel waarin de frequenties van een waarde staan die in de volledige dataset voorkomt.Meestal zijn de tabellen verticaal georiënteerd.
\end{definition}

\begin{exercise}
		De formule voor standaardafwijking en variatie staan beschreven in sectie \ref{sec:varEnSD}. 
		Hoe moet de formule aangepast worden om $\sigma$ en $\sigma^{2}$ te berekenen wanneer we te
		maken hebben met een frequentietabel? Doe dit voor de data in tabel \ref{tab:pinfreq}.
\end{exercise}

	\begin{table}
		\centering
		\caption{Aantal pinnen met frequentie}
		\label{tab:pinfreq}
		\begin{tabular}{@{}ll@{}}
			\toprule
			Pinnen $x$ & Frequentie $f_{x}$ \\ \midrule
			0          & 2                  \\
			1          & 1                  \\
			2          & 2                  \\
			3          & 0                  \\
			4          & 2                  \\
			5          & 4                  \\
			6          & 9                  \\
			7          & 11                 \\
			8          & 13                 \\
			9          & 8                  \\ \midrule
			10         & 8                  \\ \bottomrule
		\end{tabular}
	\end{table}
	
\begin{exercise}
		In de formule voor de variantie  wordt het
		verschil tussen de meetpunten en het gemiddelde gekwadrateerd.
		Waarom? Zoek dit uit door volgende berekenigswijzen eens uit te proberen.
		\[ \sigma^{2}_{1} = \frac{1}{n} \sum_{i=1}^{n} (\mu - x) \]
		\[ \sigma^{2}_{2} = \frac{1}{n} \sum_{i=1}^{n} \left| \mu - x\right| \]
		\[ \sigma^{2}_{3} = \frac{1}{n} \sum_{i=1}^{n} (\mu - x)^{2} \]
		Dataset X \[ \left\{ 4,4,-4,-4 \right\} \]
		Dataset Y \[ \left\{ 7,1,-6,-2 \right\} \]
\end{exercise}

\begin{exercise}
		Zoek eens zelfstandig op wat de variatieco\"effici\"ent is. Hoe
		wordt die gedefinieerd voor een volledige populatie en wat zou
		je ermee kunnen doen?
\end{exercise}

\section{Wetenschappelijk schrijven}
 
 In de Bachelorproef van~\textcite{Akin2016} wordt een vergelijkende studie verricht rond verschillende persistentiemogelijkheden in Android. In de Abstract kunnen we het volgende lezen:
 
 \begin{displayquote}
 	Vandaag de dag bestaan er veel applicaties, maar hoeveel daarvan blijven werken
 	zonder internetverbinding? Tegenwoordig is het ondersteunen van offline werking in
 	een applicatie geen luxe meer, maar een must-have. Om offline-support te voorzien
 	binnen een applicatie, is er nood aan het gebruik van een database. Hierdoor zijn
 	databases belangrijk binnen de IT-sector.
 	
 	Er bestaan verschillende soorten databases, maar welke moet men gebruiken?
 	Welke is het meest geschikt bij een bepaalde soort applicatie? De keuze van de database
 	kan een grote invloed hebben op verschillende eigenschappen: performantie,
 	opstartsnelheid, CPU-gebruik,.. Als de database deze eigenschappen op een negatieve
 	manier beïnvloedt, kan dit tot gevolg hebben dat het aantal gebruikers van de mobiele
 	applicatie zal verminderen. Ter beantwoording van de probleemstelling zijn volgende
 	deelvragen geformuleerd met betrekking op de applicatie:
 	\begin{itemize}
 		\item 	Wat is de invloed van de gekozen database op de opstartsnelheid? Vertraagt het
 		gebruik van de gekozen database de opstartsnelheid van de applicatie, of heeft
 		het helemaal geen invloed (in vergelijking met gebruik van andere databases)?
 		\item Wat is de invloed van de gekozen database op het CPU-gebruik? Een hoger
 		CPU-gebruik zal zorgen voor meer batterijverbruik. Zal de applicatie bij gebruik
 		van de gekozen database meer of juist minder CPU gebruiken (in vergelijking
 		met gebruik van andere databases)?
 		\item  Wat is de gemiddelde snelheid van de gekozen database bij het toevoegen van
 		records aan de database?
 	\end{itemize}
 	
 	
 	Het onderzoek werd uitgevoerd op drie verschillende applicatieprofielen: weinig data
 	(profiel 1), gemiddelde hoeveelheid data (profiel 2), veel data (profiel 3).
 	De verwachtingen waren dat Realm altijd de beste keuze zou zijn, behalve bij
 	applicatieprofiel 1. Daar zou SharedPreferences de beste keuze moeten zijn, aangezien
 	het speciaal ontwikkeld is voor kleine hoeveelheden simpele data. Het onderzoek heeft
 	echter volgend resultaat opgeleverd:
 	
 	\begin{enumerate}
 		\item Weinig data : Realm
 		\item Gemiddelde hoeveelheid data : Realm
 		\item Veel data : SQLite
 	\end{enumerate}
 	De details van het onderzoek zijn te vinden in het volgende deel van dit scriptie.
 \end{displayquote}
 
 We gaan dit onderzoek eens onder de loep nemen, kijken wat er goed aan was en wat de eventuele verbeterpunten kunnen zijn. 
 
\begin{exercise}
	Probeer volgende vragen zo goed mogelijk te beantwoorden.
	\begin{enumerate}
		\item Wat is de doelstelling van het onderzoek?
		\item Wie is het publiek?
		\item Worden de conclusies expliciet gemaakt? 
		\item Schets kort hoe de structuur van het document in elkaar zit. Komt dit overeen met wat er gezien is in de les?
	\end{enumerate}
\end{exercise}

\begin{exercise}
	Schrijf voor jezelf hoe de volgende componenten van het onderzoek ingevuld zijn.
	\begin{itemize}
		\item Context
		\item Nood
		\item Taak
		\item Object
		\item Resultaat
		\item Conclusie
		\item Perspectief
	\end{itemize}
\end{exercise}
Indien je op vorige componenten geen antwoord vindt uit de tekst, probeer dan zelf een antwoord te formuleren indien jij dit onderzoek zou uitvoeren. 

\section{Centrummaten en spreidingsmaten in R}
In vorig hoofdstuk hebben we de verschillende spreidingsmaten en centrummaten besproken. Zoals je merkt worden deze metrieken gebruikt in het onderzoek van~\textcite{Akin2016}. In de volgende oefeningen gaan we trachten de resultaten na te maken. 

Hiervoor hebben we het bestand \texttt{android\_persistence\_cpu} nodig. 

\begin{exercise}
	Open de file met excel en bekijk de structuur van het document. Hoe ziet die er uit? Kan je de variabelen identificeren en hun type benoemen. 
\end{exercise}

We gaan het programma \texttt{R} gebruiken samen met \texttt{RStudio}. Open de file in \texttt{RStudio}.

\begin{lstlisting}
	#laad de data in
	android_cpu <-  read.table("android_persistence_cpu.csv",
	header=TRUE, sep=";", na.strings="NA", dec=",", strip.white=TRUE)
	#Bekijk de data
	showData(android_cpu, placement='-20+200', font=getRcmdr('logFont'), maxwidth=80, maxheight=10)	
\end{lstlisting}

We hebben nu de data ingeladen. We kunnen eens kijken wat de gemiddelde tijd, de standaarddeviatie, de kwartielen e.a. zijn. Gebruik hiervoor de commando's \texttt{mean}, \texttt{median}, \texttt{quantile}, \texttt{min}, \texttt{max}, \texttt{var}, \texttt{sd}. Je kan ook makkelijk gebruik maken van de methode \texttt{summary}.

\begin{exercise}
	Als je de vorige metrieken berekend hebt, wat kan je daar dan over zeggen. Kan je zinnige conclusies trekken uit de vorige resultaten. Zo ja vermeld ze, zo nee beschrijf waarom je dat denkt.
\end{exercise}

\subsection{Grafieken in R}
\subsubsection{Histogram}
Een histogram is een eenvoudige plot. het toont de frequenties van de data die in een bepaald bereik voorkomen. 

\begin{lstlisting}
	#Genereert een nieuw venstertje zodat oude grafiek niet overschreven wordt
	windows();
	hist(android_cpu$Tijd,main="Verdeling van de tijd",xlab="De gemeten cpu tijd");
	windows();
	hist(android_cpu$Tijd,main="Verdeling van de tijd",xlab="De gemeten cpu tijd",breaks=2);
\end{lstlisting}
\begin{exercise}
	Wat concludeer je als je bovenstaande grafiek\footnote{Heb je wat problemen met het genereren van grafieken, volgende link \url{https://www.datacamp.com/community/tutorials/15-questions-about-r-plots\#gs.RK_ORsI} bevat een aantal goede tips and tricks om je op weg te helpen.} genereert? Is dit een zinnig resultaat? Wat gebeurt er als je de variabele breaks verhoogt?
\end{exercise}

\subsubsection{Boxplot}
Een boxplot toont de mediaan, de kwartielen, het maximum en het minimum van een dataset. Dit geeft ons een duidelijk impressie van hoe de data er uitziet.

\begin{lstlisting}
	#Genereer een nieuwe plot
	plot.new();
	boxplot(x = android_cpuTijd);
	 boxplot(android_cpu$Tijd,main='Spreiding van de CPU tijd',ylab='Tijd in ms');
\end{lstlisting} 

\begin{exercise}
	De boxplot wordt standaard verticaal getekend. Gebruik het commando \texttt{help(boxplot)} om uit te zoeken hoe we de tekening horizontaal krijgen. 
\end{exercise}

Als je goed geantwoord hebt op de volgende vragen merk je natuurlijk dat het weinig zin heeft de volledige dataset te analyseren, aangezien de dataset verdeeld is over verschillende categorie\"en. We willen dus wel deze statistieken weten, maar per categorie. We kunnen zus een boxplot maken voor elke categorie.

\begin{lstlisting}
	boxplot(android_cpu$Tijd~android_cpu$Datahoeveelheid,main='Spreiding van de CPU tijd t.o.v. datahoeveelheid',ylab='Tijd in ms');
\end{lstlisting}

\begin{exercise}
	\label{ex:boxplot}
	Interpreteer de resultaten die je behaalt uit deze grafiek. Zijn deze al wat zinniger?
\end{exercise}

We kunnen hetzelfde doen voor de verschillende soorten dataopslagmogelijkheden in android.

\begin{exercise}
	Zelfde vraag als \ref{ex:boxplot} Interpreteer de resultaten die je behaalt uit deze grafiek. Zijn deze al wat zinniger?
\end{exercise}

We kunnen eens kijken hoe de data eruit ziet over alle categorie\"en heen.

\begin{lstlisting}
	boxplot(android_cpu$Tijd~android_cpu$PersistentieType*android_cpu$Datahoeveelheid,main='Spreiding van de CPU tijd',ylab='Tijd in ms');
\end{lstlisting}



Het blijkt dat we wel al een duidelijker zicht krijgen over de data over de categorie\"en heen, maar de figuur is op dit moment te druk. 

We moeten de data dus onderverdelen in categor\"en namelijk onder \texttt{PersistentieType} en \texttt{Datahoeveelheid}. We gaan hiervoor de functie \texttt{which}\footnote{Je kan ook gebruik maken van de functie \texttt{subset}, wat misschien zelfs eenvoudiger is} gebruiken en kijken hoe de verschillende datahoeveelheden verschillen per datahoeveelheidcategorie. 

\begin{lstlisting}
	?which
	greenDOA <- android_cpu[which(android_cpu$PersistentieType=='GreenDAO'),];
	boxplot(greenDOA$Tijd~greenDOA$Datahoeveelheid);
\end{lstlisting}

\begin{exercise}
	Wat concludeer je uit de vorige grafiek?
\end{exercise}

\begin{exercise}
	Ga nu zelf na welke boxplots er interessant zijn om te maken, en kijken of jouw resultaten overeen met die van de Ozgur. Welke conclusies trek je?
\end{exercise}



