\chapter{Het onderzoeksproces}
\section{De wetenschappelijke methode}
Er zijn verschillende manieren om kennis te vergaren:

\begin{enumerate}
	\item Wetenschappelijke methode
	\item De niet-wetenschappelijke methode
\end{enumerate}

\paragraph{Niet wetenschappelijk}Er zijn verschillende versies van niet wetenschappelijk redeneren: 
\begin{description}
	\item [Autoritair] hier geldt iemand als autoriteit in een bepaald gebied en wordt als betrouwbaar bestempeld. Alles wat deze persoon beweert wordt aanzien als waarheid. 
	\item [Deductief] gegeven een set van veronderstellingen gaat men op een welbepaalde manier conclusies trekken. Alhoewel hier dus correcte conclusies kunnen behaald worden, hangt dit enkel en alleen af van de waarheid van de veronderstellingen. Maar deze veronderstellingen worden niet empirisch onderzocht.
\end{description}

\paragraph{Wetenschappelijk}
Een kenmerk van de \textsl{wetenschappelijke methode} is \textbf{Empirische validering}: gebaseerd op ervaring en directe observatie. Dus een uitspraak is geldig indien het overeen komt met wat geobserveerd wordt.


\begin{exercise}
Probeer nu vertrekkende van de niet-wetenschappelijke en wetenschappelijke manieren aan te tonen dat varkens kunnen vliegen. 
\end{exercise}


Aan de hand van zo'n empirisch onderzoek kunnen we verschillende doelen behalen:
\begin{enumerate}
	\item Exploratie: bestaat iets of gebeurt er iets?
	\item Beschrijving: wat zijn de eigenschappen van deze gebeurtenis
	\item Voorspelling: is een bepaalde gebeurtenis gerelateerd aan een andere en kan ik deze zo voorspellen?
	\item Controle: kan ik een gebeurtenis volledig voorspellen aan de hand van andere zaken?
\end{enumerate}

\paragraph{Onderzoeksdoelstellingen}

Er zijn twee grote onderzoeksdoelen die we willen behalen:

\begin{description}
	\item [Generalisatie] we gaan vaak maar een onderzoek doen op een bepaalde, beperkte groep van de totale groep (populatie). Indien we correcte conclusies kunnen trekken voor die subgroep, die ook gelden voor de totale groep dan hebben we een correcte generalisatie gevonden.
    \item[Specialisatie] Toepassen van algemene kennis op een specifiek domein of probleem. Toegepast onderzoek kan hier meestal onder geclassificeerd worden.
\end{description}

Er zijn twee soorten generalisaties
\begin{enumerate}
	\item Over 1 enkel fenomeen.
	\item Over verbanden tussen fenomenen.
\end{enumerate}
Er zijn drie redenen waarom verbanden zo belangrijk zijn:
\begin{enumerate}
	\item Volledig verstaan van een fenomeen. 
	\item Verbanden kunnen zorgen voor een voorspelling
	\item Causale verbanden: een van de fenomenen heeft dat andere fenomeen tot gevolg. 
\end{enumerate}

	
\section{Basisconcepten in onderzoek}
\paragraph{Meetniveaus}
In statistiek werken we met variabelen en waarden.

\begin{definition}[Variabele] 
    Algemene eigenschap van een object waardoor we objecten van elkaar kunnen onderscheiden. Vb.~lengte, gewicht, \ldots
\end{definition}  
\begin{definition}[Waarde]
    Specifieke eigenschap, invulling voor die variabele. Vb.~1.83m, 78 kg, \ldots
\end{definition}

Er worden meestal vier meetniveaus gebruikt in statische analyse. Het meetniveau bepaalt welke statische methodes bruikbaar zijn. 
\begin{description}
	\item [Nominaal meetniveau] \index{Nominaal}: er is slechts keuze uit een beperkt aantal categorie\"en, waarbij geen volgorde aanwezig is tussen de antwoorden.
	\item [Ordinaal meetniveau] \index{Ordinaal}: een variabele die is ingedeeld in categorie\"en, waar er echter wel een logische volgorde is tussen de categori\"en. 
	\item [Intervalniveau] \index{Intervalniveau}: variabelen die niet in categorie\"en voorkomen, en waarbij berekeningen kunnen mee uitgevoerd worden, maar zonder nulpunt.
	\item [Rationiveau] \index{Rationiveau}: intervalniveau met nulpunt. Je kunt hierdoor verhoudingen berekenen tussen verschillende waarden op de schaal.
\end{description}

\begin{exercise}
	Zoek zelf nu eens voorbeelden voor de verschillende meetniveaus.
\end{exercise}

\paragraph{Onderzoeksproces}
Het onderzoeksproces kan grotendeels opgedeeld worden in 6 grote delen:
\begin{enumerate}
	\item Formuleren van de probleemstelling: wat is de onderzoeksvraag
	\item Exacte informatiebehoefte defini\"eren: welke specifieke vragen moeten we stellen
	\item Uitvoeren van het onderzoek: enqu\^etes, simulaties, \dots
	\item Verwerken van de gegevens: statistische software
	\item Analyseren van de gegevens: uitvoeren van de statistische methodes
	\item Conclusies schrijven: schrijven van onderzoeksverslag
\end{enumerate}

\begin{definition}[Oorzakelijk verband]
\index{Oorzakelijk verband} Een variabele veroorzaakt een oorzakelijk verband wanneer een verandering in die variabele op een betrouwbare manier een geassocieerde verandering van een andere variabele tot gevolg heeft, op voorwaarde dat alle andere potenti\"ele oorzaken ge\"elimineerd zijn.
\end{definition}

Er is niet altijd verband zichtbaar en we moeten soms verder kijken dan naar de absolute waarden van de variabelen alvorens conclusies te trekken.

\begin{example}
	Bij het voobeeld van Pepsi versus cola zou je initeel kunnen denken dat Pepsi lekkerder is omdat er meer mensen ervan geproefd hebben (70 ten opzichte van 30). Maar dit zou een verkeerde manier van redeneren zijn. We moeten relatief ten op zichte van de  \textit{marginale} totalen kijken. Hier zien we dan dat 56 van de 70 ($\frac{56}{70} = 0.8$) mensen die Pepsi gedronken hebben het lekker vonden en 24 van de 30 ($\frac{24}{30}$=0.8) mensen vonden cola lekker. Dus is er geen verschil in waarden voor Cola en Pepsi voor de gemiddelde waarde van smaak. 
\end{example}
