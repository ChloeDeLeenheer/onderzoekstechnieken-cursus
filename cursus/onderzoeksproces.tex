\chapter{Het onderzoeksproces}
\label{ch:onderzoeksproces}

\section{De wetenschappelijke methode}

Er zijn verschillende manieren om kennis te vergaren:

\begin{enumerate}
	\item wetenschappelijke methode, maar ook 
	\item een niet-wetenschappelijke methode
\end{enumerate}

\paragraph{Niet-wetenschappelijk}Er zijn verschillende versies van niet-wetenschappelijk redeneren: 
\begin{description}
	\item [Autoritair] hier geldt iemand als autoriteit in een bepaald gebied en wordt als betrouwbaar bestempeld. Alles wat deze persoon beweert wordt aanzien als waarheid. 
	\item [Deductief] gegeven een set van veronderstellingen gaat men op een welbepaalde manier conclusies trekken. Alhoewel hier dus correcte conclusies kunnen behaald worden, hangt dit enkel en alleen af van de waarheid van de veronderstellingen. Maar deze veronderstellingen worden niet-empirisch onderzocht.
\end{description}

\paragraph{Wetenschappelijk}
Een kenmerk van de \textsl{wetenschappelijke methode} is \textbf{empirische validering}: gebaseerd op ervaring en directe observatie. Dus een uitspraak is geldig indien het overeenkomt met wat geobserveerd wordt.

\begin{exercise}
Probeer nu vertrekkende van de niet-wetenschappelijke en wetenschappelijke manieren aan te tonen dat varkens kunnen vliegen. 
\end{exercise}

Aan de hand van zo'n empirisch onderzoek kunnen we verschillende doelen behalen:
\begin{enumerate}
	\item Exploratie: bestaat iets of gebeurt er iets?
	\item Beschrijving: wat zijn de eigenschappen van deze gebeurtenis?
	\item Voorspelling: is een bepaalde gebeurtenis gerelateerd aan een andere en kan ik deze zo voorspellen?
	\item Controle: kan ik een gebeurtenis volledig voorspellen aan de hand van andere zaken?
\end{enumerate}


\paragraph{Onderzoeksdoelstellingen}

Er zijn twee grote onderzoeksdoelen die we willen behalen:

\begin{description}
  \item [Generalisatie] we gaan vaak maar een onderzoek doen op een bepaalde, beperkte groep van de totale groep (populatie). Indien we correcte conclusies kunnen trekken voor die subgroep, die ook gelden voor de totale groep, dan hebben we een correcte generalisatie gevonden.
  \item[Specialisatie] Toepassen van algemene kennis op een specifiek domein of probleem. Toegepast onderzoek kan hier meestal onder geclassificeerd worden.
\end{description}

Er zijn twee soorten generalisaties:
\begin{enumerate}
	\item Over 1 enkel fenomeen.
	\item Over verbanden tussen fenomenen.
\end{enumerate}
Er zijn drie redenen waarom verbanden zo belangrijk zijn:
\begin{enumerate}
	\item Volledig verstaan van een fenomeen. 
	\item Verbanden kunnen zorgen voor een voorspelling
	\item Causale verbanden: één van de fenomenen heeft dat andere fenomeen tot gevolg. 
\end{enumerate}

\paragraph{Fundamenteel vs.~toegepast onderzoek}

Afhankelijk van de onderzoeksdoelstelling spreken we van hetzij fundamenteel, hetzij toegepast onderzoek.

\emph{Fundamenteel onderzoek} wordt typisch aan universiteiten uitgevoerd. Onderzoekers trachten de bestaande kennis in hun vakgebied uit te breiden. In computerwetenschappen kan het bijvoorbeeld gaan over het ontwikkelen van nieuwe algoritmen. Bij fundamenteel onderzoek wordt er niet in de eerste plaats rekening gehouden met de praktische toepassingen. Je zou kunnen zeggen dat hier in de eerste plaats geprobeerd wordt om oplossingsmethoden te ontwikkelen, en pas dan gekeken wordt welke problemen er efficiënt(er) kunnen mee opgelost worden. Het is moeilijk a priori te voorspellen welke impact (in het bijzonder financiële meerwaarde) fundamentele onderzoeksresultaten kunnen hebben. In het beste geval kunnen ze de wereld veranderen, in het slechtste komt er geen enkele praktische toepassing.

\emph{Toegepast onderzoek} begint bij een concreet probleem, typisch in een bedrijfscontext. Onderzoekers moeten zich eerst inwerken in het specifieke probleemdomein. Dan kunnen ze op zoek gaan naar de meest geschikte methode om dat probleem op te lossen. Daarom moeten ze ook op de hoogte zijn van de state-of-the-art binnen het relevante fundamentele onderzoek. De meerwaarde van toegepast onderzoek is meestal makkelijker te meten, maar de impact blijft beperkt tot het bedrijf/organisatie in wiens opdracht het onderzoek werd uitgevoerd.

\section{Basisconcepten in onderzoek}

\paragraph{Meetniveaus}

In statistiek werken we met variabelen en waarden.

\begin{definition}[Variabele] 
    Algemene eigenschap van een object waardoor we objecten van elkaar kunnen onderscheiden. Vb.~lengte, gewicht, \ldots
\end{definition}  
\begin{definition}[Waarde]
    Specifieke eigenschap, invulling voor die variabele. Vb.~1.83m, 78 kg, \ldots
\end{definition}

Er worden meestal vier meetniveaus gebruikt in statistische analyse. Het meetniveau bepaalt welke statische methodes bruikbaar zijn. 
\begin{description}
	\item [Nominaal meetniveau] \index{Nominaal}: er is slechts keuze uit een beperkt aantal categorie\"en, waarbij geen volgorde aanwezig is tussen de antwoorden.
	\item [Ordinaal meetniveau] \index{Ordinaal}: een variabele die is ingedeeld in categorie\"en, waar er echter wel een logische volgorde is tussen de categori\"en. 
	\item [Intervalniveau] \index{Intervalniveau}: variabelen die niet in categorie\"en voorkomen, en waarbij berekeningen kunnen mee uitgevoerd worden, maar zonder nulpunt.
	\item [Rationiveau] \index{Rationiveau}: intervalniveau met nulpunt. Je kunt hierdoor verhoudingen berekenen tussen verschillende waarden op de schaal.
\end{description}

\begin{exercise}
	Zoek zelf nu eens voorbeelden voor de verschillende meetniveaus.
\end{exercise}

\paragraph{Onderzoeksproces}
Het onderzoeksproces kan grotendeels opgedeeld worden in 6 grote fasen:
\begin{enumerate}
	\item Formuleren van de probleemstelling: wat is de onderzoeksvraag?
	\item Exacte informatiebehoefte defini\"eren: welke specifieke vragen moeten we stellen?
	\item Uitvoeren van het onderzoek: enqu\^etes, simulaties, \dots
	\item Verwerken van de gegevens: statistische software
	\item Analyseren van de gegevens: uitvoeren van de statistische methodes
	\item Conclusies schrijven: schrijven van onderzoeksverslag
\end{enumerate}

\begin{definition}[Oorzakelijk verband]
\index{Oorzakelijk verband} Een variabele veroorzaakt een oorzakelijk verband wanneer een verandering in die variabele op een betrouwbare manier een geassocieerde verandering van een andere variabele tot gevolg heeft, op voorwaarde dat alle andere potenti\"ele oorzaken ge\"elimineerd zijn.
\end{definition}

Er is niet altijd een verband zichtbaar, en we moeten soms verder kijken dan naar de absolute waarden van de variabelen alvorens conclusies te trekken.

\begin{example}
	Bij het voobeeld van Pepsi versus cola zou je initeel kunnen denken dat Pepsi lekkerder is omdat er meer mensen ervan geproefd hebben (70 ten opzichte van 30). Maar dit zou een verkeerde manier van redeneren zijn. We moeten relatief ten op zichte van de  \textit{marginale} totalen kijken. Hier zien we dan dat 56 van de 70 ($\frac{56}{70} = 0.8$) mensen die Pepsi gedronken hebben het lekker vonden, en 24 van de 30 ($\frac{24}{30}$=0.8) mensen vonden cola lekker. Dus is er geen verschil in waarden voor cola en Pepsi voor de gemiddelde waarde van smaak.
\end{example}

\section{Oefeningen}
\label{sec:proces-oefeningen}


In de Bachelorproef van~\textcite{Akin2016} wordt een vergelijkende studie verricht rond verschillende persistentiemogelijkheden in Android. In de Abstract kunnen we het volgende lezen:

\begin{displayquote}
  Vandaag de dag bestaan er veel applicaties, maar hoeveel daarvan blijven werken
  zonder internetverbinding? Tegenwoordig is het ondersteunen van offline werking in
  een applicatie geen luxe meer, maar een must-have. Om offline-support te voorzien
  binnen een applicatie, is er nood aan het gebruik van een database. Hierdoor zijn
  databases belangrijk binnen de IT-sector.
  
  Er bestaan verschillende soorten databases, maar welke moet men gebruiken?
  Welke is het meest geschikt bij een bepaalde soort applicatie? De keuze van de database
  kan een grote invloed hebben op verschillende eigenschappen: performantie,
  opstartsnelheid, CPU-gebruik,.. Als de database deze eigenschappen op een negatieve
  manier beïnvloedt, kan dit tot gevolg hebben dat het aantal gebruikers van de mobiele
  applicatie zal verminderen. Ter beantwoording van de probleemstelling zijn volgende
  deelvragen geformuleerd met betrekking op de applicatie:
  \begin{itemize}
    \item 	Wat is de invloed van de gekozen database op de opstartsnelheid? Vertraagt het
    gebruik van de gekozen database de opstartsnelheid van de applicatie, of heeft
    het helemaal geen invloed (in vergelijking met gebruik van andere databases)?
    \item Wat is de invloed van de gekozen database op het CPU-gebruik? Een hoger
    CPU-gebruik zal zorgen voor meer batterijverbruik. Zal de applicatie bij gebruik
    van de gekozen database meer of juist minder CPU gebruiken (in vergelijking
    met gebruik van andere databases)?
    \item  Wat is de gemiddelde snelheid van de gekozen database bij het toevoegen van
    records aan de database?
  \end{itemize}
  
  
  Het onderzoek werd uitgevoerd op drie verschillende applicatieprofielen: weinig data
  (profiel 1), gemiddelde hoeveelheid data (profiel 2), veel data (profiel 3).
  De verwachtingen waren dat Realm altijd de beste keuze zou zijn, behalve bij
  applicatieprofiel 1. Daar zou SharedPreferences de beste keuze moeten zijn, aangezien
  het speciaal ontwikkeld is voor kleine hoeveelheden simpele data. Het onderzoek heeft
  echter volgend resultaat opgeleverd:
  
  \begin{enumerate}
    \item Weinig data : Realm
    \item Gemiddelde hoeveelheid data : Realm
    \item Veel data : SQLite
  \end{enumerate}
  De details van het onderzoek zijn te vinden in het volgende deel van dit scriptie.
\end{displayquote}

We gaan dit onderzoek eens onder de loep nemen, kijken wat er goed aan was en wat de eventuele verbeterpunten kunnen zijn. 

\begin{exercise}
  Probeer volgende vragen zo goed mogelijk te beantwoorden.
  \begin{enumerate}
    \item Wat is de doelstelling van het onderzoek?
    \item Wie is het publiek?
    \item Worden de conclusies expliciet gemaakt? 
    \item Schets kort hoe de structuur van het document in elkaar zit. Komt dit overeen met wat er gezien is in de les?
  \end{enumerate}
\end{exercise}

\begin{exercise}
  Schrijf voor jezelf hoe de volgende componenten van het onderzoek ingevuld zijn.
  \begin{itemize}
    \item Context
    \item Nood
    \item Taak
    \item Object
    \item Resultaat
    \item Conclusie
    \item Perspectief
  \end{itemize}
  
  Indien je op vorige componenten geen antwoord vindt uit de tekst, probeer dan zelf een antwoord te formuleren indien jij dit onderzoek zou uitvoeren. 
\end{exercise}
