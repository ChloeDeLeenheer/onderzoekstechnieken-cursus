\chapter{Het onderzoeksproces}
\label{ch:onderzoeksproces}

\section{Leerdoelen}
\label{sec:onderzoeksproces-leerdoelen}

Na dit hoofdstuk moet je in staat zijn om:

\begin{itemize}
  \item De begrippen in dit hoofdstuk te defini\"eren;
    \begin{itemize}
      \item wetenschappelijke methode en empirische validering
      \item fundamenteel en toegepast onderzoek
      \item variabele, waarde
      \item meetniveau: nominaal, ordinaal, interval, ratio
      \item populatie, steekproef, steekproefkader
      \item aselecte vs.~selecte steekproef
      \item gestratificeerde steekproef
      \item steekproeffouten: toevallig/systematisch, steekproeffout/niet-steekproeffout
    \end{itemize}
  \item Uit de beschrijving van een casus of situatie te bepalen of ze al dan niet voldoen aan de kenmerken van de wetenschappelijke methode of empirische validering;
  \item Doelstellingen van wetenschappelijke onderzoek te benoemen en uit te leggen;
  \item De verschillende stappen in een onderzoeksproces te benoemen en uit te leggen;
  \item De vier meetniveaus op te sommen, de kenmerken van elk te formuleren en een voorbeeld te geven;
  \item Voor een gegeven variabele het meetniveau te bepalen;
  \item De types steekproeffouten op te sommen, de kenmerken van elk te formuleren en een voorbeeld te geven;
  \item Uit een beschrijving van een steekproefmethode:
    \begin{itemize}
      \item een onderscheid te maken tussen een aselecte en selecte steekproefmethode;
      \item het type steekproeffout(en) dat gemaakt werd te benoemen en uit te leggen;
      \item voorstellen tot verbetering van de steekproefmethode voor te stellen;
    \end{itemize}
\end{itemize}

\section{De wetenschappelijke methode}

Er zijn verschillende manieren om kennis te vergaren:

\begin{enumerate}
  \item wetenschappelijke methode, maar ook 
  \item een niet-wetenschappelijke methode
\end{enumerate}

\paragraph{Niet-wetenschappelijk}Er zijn verschillende versies van niet-wetenschappelijk redeneren: 
\begin{description}
  \item [Autoritair] hier geldt iemand als autoriteit in een bepaald gebied en wordt als betrouwbaar bestempeld. Alles wat deze persoon beweert wordt aanzien als waarheid. 
  \item [Deductief] gegeven een set van veronderstellingen gaat men op een welbepaalde manier conclusies trekken. Alhoewel hier dus correcte conclusies kunnen behaald worden, hangt dit enkel en alleen af van de waarheid van de veronderstellingen. Deze veronderstellingen worden echter niet-empirisch onderzocht.
\end{description}

\paragraph{Wetenschappelijk}
Een kenmerk van de \textsl{wetenschappelijke methode} is \textbf{empirische validering}: gebaseerd op ervaring en directe observatie. Een uitspraak is dus geldig indien deze overeenkomt met wat geobserveerd wordt.

\begin{exercise}
Probeer nu vertrekkende van de niet-wetenschappelijke en wetenschappelijke manieren aan te tonen dat varkens kunnen vliegen. 
\end{exercise}

Aan de hand van zo'n empirisch onderzoek kunnen we verschillende doelen behalen:
\begin{enumerate}
  \item Exploratie: bestaat iets of gebeurt er iets?
  \item Beschrijving: wat zijn de eigenschappen van deze gebeurtenis?
  \item Voorspelling: is een bepaalde gebeurtenis gerelateerd aan een andere en kan ik deze zo voorspellen?
  \item Controle: kan ik een gebeurtenis volledig voorspellen aan de hand van andere zaken?
\end{enumerate}


\paragraph{Onderzoeksdoelstellingen}

Er zijn twee grote onderzoeksdoelen die we willen behalen:

\begin{description}
  \item [Generalisatie] we gaan vaak maar een onderzoek doen op een bepaalde, beperkte groep van de totale groep (populatie). Indien we correcte conclusies kunnen trekken voor die subgroep, die ook gelden voor de totale groep, dan hebben we een correcte generalisatie gevonden.
  \item[Specialisatie] Toepassen van algemene kennis op een specifiek domein of probleem. Toegepast onderzoek kan hier meestal onder geclassificeerd worden.
\end{description}

Er zijn twee soorten generalisaties:
\begin{enumerate}
  \item Over 1 enkel fenomeen.
  \item Over verbanden tussen fenomenen.
\end{enumerate}
Er zijn drie redenen waarom verbanden zo belangrijk zijn:
\begin{enumerate}
  \item Volledig verstaan van een fenomeen. 
  \item Verbanden kunnen zorgen voor een voorspelling.
  \item Causale verbanden: één van de fenomenen heeft dat andere fenomeen tot gevolg. 
\end{enumerate}

\paragraph{Fundamenteel vs.~toegepast onderzoek}

Afhankelijk van de onderzoeksdoelstelling spreken we van hetzij fundamenteel, hetzij toegepast onderzoek.

\emph{Fundamenteel onderzoek} wordt typisch aan universiteiten uitgevoerd. Onderzoekers trachten de bestaande kennis in hun vakgebied uit te breiden. In computerwetenschappen kan het bijvoorbeeld gaan over het ontwikkelen van nieuwe algoritmen. Bij fundamenteel onderzoek wordt er niet in de eerste plaats rekening gehouden met de praktische toepassingen. Je zou kunnen zeggen dat hier in de eerste plaats geprobeerd wordt om oplossingsmethoden te ontwikkelen, en pas dan gekeken wordt welke problemen er effici\"ent(er) kunnen mee opgelost worden. Het is moeilijk a priori te voorspellen welke impact (in het bijzonder financi\"ele meerwaarde) fundamentele onderzoeksresultaten kunnen hebben. In het beste geval kunnen ze de wereld veranderen, in het slechtste komt er geen enkele praktische toepassing.

\emph{Toegepast onderzoek} begint bij een concreet probleem, typisch in een bedrijfscontext. Onderzoekers moeten zich eerst inwerken in het specifieke probleemdomein. Dan kunnen ze op zoek gaan naar de meest geschikte methode om dat probleem op te lossen. Daarom moeten ze ook op de hoogte zijn van de state-of-the-art binnen het relevante fundamentele onderzoek. De meerwaarde van toegepast onderzoek is meestal makkelijker te meten, maar de impact blijft beperkt tot het bedrijf of de organisatie waarvoor het onderzoek werd uitgevoerd.

\section{Basisconcepten in onderzoek}

\paragraph{Meetniveaus}

Binnen de statistiek werken we met variabelen en waarden.

\begin{definition}[Variabele] 
  Algemene eigenschap van een object waardoor we objecten van elkaar kunnen onderscheiden. Vb.~lengte, gewicht, \ldots
\end{definition}  
\begin{definition}[Waarde]
  Specifieke eigenschap, invulling voor die variabele. Vb.~1.83m, 78 kg, \ldots
\end{definition}

Er worden meestal vier meetniveaus gebruikt in statistische analyse. Het meetniveau bepaalt welke statische methodes bruikbaar zijn.

\begin{description}
  \item [Nominaal meetniveau] \index{Nominaal}: er is slechts keuze uit een beperkt aantal categorie\"en, waarbij geen volgorde aanwezig is tussen de antwoorden.
  \item [Ordinaal meetniveau] \index{Ordinaal}: een variabele die is ingedeeld in categorie\"en, waar er echter wel een logische volgorde is tussen de categorie\"en. 
  \item [Intervalniveau] \index{Intervalniveau}: variabelen die als waarde een (re\"eel) getal hebben, en waar berekeningen kunnen mee uitgevoerd worden, maar zonder nulpunt.
  \item [Rationiveau] \index{Rationiveau}: intervalniveau met nulpunt. Hierdoor kan je verhoudingen berekenen tussen verschillende waarden op de schaal.
\end{description}

\paragraph{Onderzoeksproces}
Het onderzoeksproces kan grotendeels opgedeeld worden in 6 grote fasen:
\begin{enumerate}
  \item Formuleren van de probleemstelling: wat is de onderzoeksvraag?
  \item Exacte informatiebehoefte defini\"eren: welke specifieke vragen moeten we stellen?
  \item Uitvoeren van het onderzoek: enqu\^etes, simulaties, \dots
  \item Verwerken van de gegevens: statistische software
  \item Analyseren van de gegevens: uitvoeren van de statistische methodes
  \item Conclusies schrijven: schrijven van onderzoeksverslag
\end{enumerate}

\begin{definition}[Oorzakelijk verband]
\index{Oorzakelijk verband} Een variabele veroorzaakt een oorzakelijk verband wanneer een verandering in die variabele op een betrouwbare manier een geassocieerde verandering van een andere variabele tot gevolg heeft, op voorwaarde dat alle andere potenti\"ele oorzaken ge\"elimineerd zijn.
\end{definition}

Er is niet altijd een verband zichtbaar, en we moeten soms verder kijken dan naar de absolute waarden van de variabelen alvorens conclusies te trekken.

\begin{example}
  Bekijk de waarden in Tabel~\ref{tab:pepsi-coca}, die het resultaat zijn van een smaaktest. Aan honderd mensen werd gevraagd om blind \'e\'en soort cola te proeven en dan te zeggen of ze het al dan niet lekker vonden.
  
  Initieel zou je kunnen denken dat Pepsi lekkerder is omdat er meer mensen dit lekkerder vonden (56 t.o.v. 24 voor Coca Cola). Dit zou echter een verkeerde manier van redeneren zijn, er zijn immers meer mensen die Pepsi geproefd hebben. We moeten dus relatief ten opzichte van de \textit{marginale} totalen kijken. Hier zien we dan dat 56 van de 70 mensen ($\frac{56}{70} = 0.8$) die Pepsi gedronken hebben het lekker vonden, en 24 van de 30 mensen ($\frac{24}{30} =0.8$) vonden cola lekker. Er is dus in feite geen verschil in de beoordeling van beide merken.
\end{example}

\begin{table}
  \centering
  \begin{tabular}{l|cc|c}
  	            & Pepsi & Coca Cola & Totaal \\
  	\midrule
  	Lekker      &  56   &    24     &   80   \\
  	Niet lekker &  14   &     6     &   20   \\
  	\midrule
  	Totaal      &  70   &    30     &  100
  \end{tabular}
  \caption[Resultaten van een smaaktest tussen Pepsi en Coca Cola.]{Resultaten van een smaaktest tussen Pepsi en Coca Cola. Aan honderd mensen werd gevraagd om blind een soort cola te proeven (hetzij Pepsi, hetzij Coca Cola) en te zeggen of ze het al dan niet lekker vonden.}
  \label{tab:pepsi-coca}
\end{table}

\section{Steekproefonderzoek}
\label{sec:steekproefonderzoek}

Een reden om kwantitatief onderzoek uit te voeren is het kunnen doen van uitspraken die een representatief beeld van de werkelijkheid geven. Hierbij wordt vaak gebruik gemaakt van een steekproef. Een steekproef is een selectie uit een totale populatie ten behoeve van een meting van bepaalde eigenschappen van die populatie.

\subsection{Populatie en Steekproeven}

\begin{definition}[Populatie]
  De verzameling van \textbf{alle} elementen (objecten, personen, respondenten, \ldots) waar men onderzoek naar wil doen, noemt men de \index{populatie}\emph{populatie}. Een ander woord is ook wel de doelgroep.
\end{definition}

\begin{definition}[Steekproefkader]
  Een \index{steekproefkader}steekproefkader is een lijst die alle leden van een te onderzoeken populatie opsomt.
\end{definition}

\begin{definition}[Steekproef]
  Een \index{steekproef}\emph{steekproef} is een deelverzameling van de populatie waar de onderzoekers effectief metingen op zullen uitvoeren, of waarover ze specifieke informatie zullen verzamelen.
\end{definition}

\begin{figure}
  \begin{center}
    \begin{tikzpicture}[scale=.55]
      \fill[hgyellow] (2,2) ellipse (4cm and 2cm) ;
      \fill[hgorange] (1.5,2) ellipse (2cm and 1cm) ;
      \node[draw=none,minimum size=1cm,inner sep=0pt] at (3,0.5) {populatie};
      \node[draw=none,minimum size=1cm,inner sep=0pt] at (2.5,2) {steekproef};
    \end{tikzpicture}
  \end{center}
  \caption{Populatie en steekproef}
  \label{img:populatie-steekproef}
\end{figure}

Er zijn een aantal redenen waarom een steekproef genomen wordt:

\begin{itemize}
  \item Populatie is te groot om de nodige informatie over de gehele populatie te verzamelen.
  \item Het uitvoeren van een meting is te duur, waardoor het onderzoek te veel zou gaan kosten.
  \item Wanneer snelheid belangrijk is, is het vaak sneller een subgroep te onderzoeken.
  \item Het nemen van een steekproef is sowieso makkelijker dan de gehele populatie te onderzoeken.
  \item \dots
\end{itemize}

Na het nemen van de nodige metingen, of het verzamelen van de nodige informatie, zullen de onderzoekers een conclusie trekken. Onder bepaalde strikte voorwaarden mogen de resultaten voor de steekproef ook veralgemeend worden tot de gehele populatie. Met andere woorden, als je de steekproef correct genomen hebt, mag je veronderstellen dat wat je observeert binnen de steekproef zal gelden voor de gehele populatie. Verderop in de cursus gaan we dieper in op wat deze voorwaarden precies zijn.

\subsection{Kiezen van steekproefmethode}

\subsubsection{Aselecte steekproef}

Er zijn verschillende technieken om een steekproef te nemen uit een populatie. In een ideale wereld verloopt de beste manier om een steekproef op te zetten als volgt:

\begin{enumerate}
  \item \textbf{Definitie populatie}: wat is de precieze doelgroep? Dit hangt nauw samen met de probleemstelling van het onderzoek. Dit is een zeer belangrijke stap waar je niet licht over mag gaan. Het is belangrijk de doelgroep/populatie zo goed mogelijk te beschrijven. Elementen die van belang zijn, zijn bijvoorbeeld sociale, demografische of fysieke kenmerken zoals geslacht, leeftijd, woonplaats, \dots
  \item \textbf{Bepalen van steekproefkader}: als de doelgroep goed omschreven is, dan kan je een lijst opstellen van alle elementen die deel uitmaken van de populatie. Dat kunnen personen zijn, bedrijven, producten, enz., afhankelijk van de onderzoeksvraag.
  \item \textbf{Willekeurige steekproef}: vervolgens selecteren de onderzoekers aan de hand van het steekproefkader \textit{willekeurig} een aantal elementen uit waarover ze informatie zullen verzamelen.
\end{enumerate}

Het is enorm belangrijk dat het selectieproces effectief willekeurig is, d.w.z.~dat elk element van de populatie een even grote kans heeft om gekozen te worden. De reden hiervoor zullen we later in deze cursus naderbij onderzoeken, wanneer we het hebben over de Centrale Limietstelling (zie Sectie~\ref{sec:centrale-limietstelling}).

\begin{definition}[aselecte steekproef]
  Een steekproef waar elk element een even grote kans heeft om geselecteerd te worden, noemen we \index{steekproef!aselecte}\emph{aselect}.
\end{definition}

Jammer genoeg is het vaak onpraktisch om een aselecte steekproef te nemen. Soms is het praktisch onmogelijk om een steekproefkader op te stellen, ook al is de doelgroep goed gedefinieerd.

\begin{example}
  Elk jaar voert het onderzoekscentrum imec in Vlaanderen een onderzoek rond het gebruik van media en technologie. De populatie is gedefinieerd als alle Vlamingen van minstens 16 jaar.
  
  Op 1 januari 2019 telde het Vlaamse Gewest ongeveer 6,6 miljoen inwoners, Nederlandstaligen in Brussel niet meegerekend~\autocite{Statbel2019}. Je kan je wel voorstellen dat het als onderzoeksinstelling niet mogelijk is om een lijst te bekomen met de namen en contactgegevens van al deze inwoners. De overheid is wellicht de enige instantie die in staat is om zo'n lijst op te stellen, maar de wet op de bescherming van de persoonlijke levenssfeer laat niet toe dat die zomaar gedeeld wordt met andere organisaties.
  
  Om die reden past imec noodgedwongen een andere methode toe om een steekproef te bepalen.
\end{example}

\subsubsection{Gestratificeerde steekproef}

Soms is de populatie die men wenst te bestuderen erg verschillend op een aantal belangrijke kenmerken. Daarom wordt de populatie als geheel in een aantal elkaar niet-overlappende en homogene strata of klassen ingedeeld, bijvoorbeeld leeftijdscategorie\"en, geslacht, opleidingsniveau, enz.

\begin{definition}[Gestratificeerde steekproef]
  \index{steekproef!gestratificeerde}Een \textbf{gestratificeerde}  steekproef is een steekproef waarbij het aandeel van de subpopulatie in de steekproef gelijk is aan het aandeel van de subpopulatie in de populatie als geheel.
\end{definition}

\begin{example}
  Indien we uit een populatie kijken naar de leeftijd van  mannen en vrouwen, zien we in Tabel~\ref{tab:frequenties-populatie} de absolute waarden. We kunnen niet alle leden van de populatie ondervragen, maar indien we een steekproef nemen waarbij de mannen en leeftijdscategorie\"en relatief equivalent zijn met de populatie, hebben we een gestratificeerde steekproef genomen (zie Tabel~\ref{tab:frequenties-steekproef}).
\end{example}

\begin{table}
  \centering
  \begin{tabular}{l|cccc|c}
    & \multicolumn{4}{c|}{\textbf{Leeftijd}} & \\
    Geslacht & $\le 18$ & $]18,25]$ & $]25, 40]$ & $> 40$ & Totaal\\
    \hline
    Vrouw & 500 & 1500 & 1000 & 250 & 3250 \\
    Man   & 400 & 1200 & 800 & 160 & 2560\\
    \hline
    Totaal & 900 & 2700 & 1800 & 410 & 5810
  \end{tabular}
  \caption{Frequenties binnen een bepaalde fictieve populatie volgens geslacht en leeftijdscategorie}
  \label{tab:frequenties-populatie}
\end{table}

\begin{table}
  \centering
  \begin{tabular}{l|cccc|c}
    & \multicolumn{4}{c|}{\textbf{Leeftijd}} & \\
    Geslacht & $\le 18$ & $]18,25]$ & $]25, 40]$ & $> 40$ & Totaal\\
    \hline
    Vrouw & 50 & 150 & 100 & 25 & 325 \\
    Man   & 40 & 120 & 80 & 16 & 256\\
    \hline
    Totaal & 90 & 270 & 180 & 41 & 581
  \end{tabular}
  \caption{Steekproef gestratificeerd volgens geslacht en leeftijdscategorie.}
  \label{tab:frequenties-steekproef}
\end{table}

Het voordeel van een gestratificeerde steekproef is dat je beter kan controleren of de steekproef representatief is voor de populatie als geheel. Omdat de onderzoekers hier een selectie maken op basis van specifieke eigenschappen, kan ze echter niet als aselect beschouwd worden.

Nadat gestratificieerd is, moet bepaald worden op welke wijze binnen ieder stratum het aantal benodigde objecten of respondenten gekozen moet worden. Indien mogelijk gebeurt dit best op een aselecte manier.

%\subsubsection{Andere steekproefmethoden}

% TODO: Eventueel andere steekproefmethoden opsommen, bv. ahv
% https://www.studiemeesters.nl/scriptie/steekproefmethode-steekproef-nemen-doe-je-zo/

\subsection{Fouten bij steekproeven}

Wat je ook doet om de steekproef zo goed mogelijk te nemen, je metingen zullen nagenoeg altijd verschillen van wat je zou bekomen als je de gehele populatie onderzoekt. Er zitten dus sowieso \emph{fouten} in de resultaten. Deze fouten kunnen onderverdeeld worden in enkele categorie\"en:

\paragraph{Toevallige steekproeffouten}

Ook als je een aselecte steekproef neemt, is het mogelijk dat bij de geselecteerde elementen de metingen afwijken van wat in de populatie als geheel zou gemeten worden. In dit geval spreken we van een \emph{toevallige steekproeffout}. Dit soort fouten is onvermijdelijk.

\paragraph{Systematische steekproeffouten}

Een procedure in de steekproef die een fout oplevert die een systematische oorzaak heeft en dus niet te wijten is aan toevallige effecten. Bijvoorbeeld door systematisch een bevoordeeld deel van de populatie te ondervragen. Als je een online enqu\^ete voert, dan sluit je iedereen uit die geen internetverbinding heeft. Ook gaan mensen eerder deelnemen aan een enquête als ze een zekere affiniteit hebben met het onderwerp.

\begin{example}
  In de wetenschappelijke literatuur, ook in toptijdschriften, worden regelmatig algemene uitspraken gedaan over menselijke psychologie en gedrag die gebaseerd zijn op steekproeven die volledig genomen zijn uit westerse, hoog opgeleide, ge\"industrialiseerde, rijke en democratische samenlevingen. In het Engels worden deze eigenschappen (Western, Educated, Industrialized, Rich, en Democratic) afgekort tot \emph{WEIRD}.
  
  De wetenschappers gaan er van uit dat deze groepen representatief zijn voor de wereldbevolking als geheel, maar dat is een gevaarlijke veronderstelling. Een studie van \textcite{HenrichEtAl2010} concludeert dat er belangrijke aanwijzingen zijn dat de leden van \emph{WEIRD}e samenlevingen ongewone eigenschappen vertonen ten opzichte van de wereldbevolking als geheel, en soms zelfs als uitschieters kunnen beschouwd worden!
  
  Hierdoor komen de conclusies uit dit soort onderzoeken op losse schroeven te staan!
\end{example}

\paragraph{Toevallige niet-steekproeffouten}

Een niet-steekproeffout heeft niet te maken met de manier waarop de steekproef is opgezet, maar eerder met de meting of het verzamelen van informatie binnen de steekproef.

Onder \emph{toevallige niet-steekproeffouten} vallen bijvoorbeeld verkeerd aangekruiste antwoorden, hetzij per ongeluk, hetzij omdat de respondent de vraag anders heeft ge\"interpreteerd dan de onderzoeker bedoelde.

Dit soort fouten zijn te vermijden door bijvoorbeeld de vragen duidelijk en niet verkeerd interpreteerbaar te formuleren, door een strikte meetprocedure te ontwerpen en toe te passen, enz.

\subsubsection{Systematische niet-steekproeffouten}

Wanneer bijvoorbeeld respondenten met een sterke band met het onderzoek eerder geneigd zijn om een vragenlijst in te vullen, ga je positievere antwoorden krijgen - terwijl ze niet representatief zijn voor de gehele populatie.

\section{Oefeningen}
\label{sec:proces-oefeningen}

\subsection{Onderzoeksproces}

In de Bachelorproef van~\textcite{Akin2016} wordt een vergelijkende studie verricht rond verschillende persistentiemogelijkheden in Android. In de Abstract kunnen we het volgende lezen:

\begin{displayquote}
  \itshape
  Vandaag de dag bestaan er veel applicaties, maar hoeveel daarvan blijven werken zonder internetverbinding? Tegenwoordig is het ondersteunen van offline werking in een applicatie geen luxe meer, maar een must-have. Om offline-support te voorzien binnen een applicatie, is er nood aan het gebruik van een database. Hierdoor zijn databases belangrijk binnen de IT-sector.
  
  Er bestaan verschillende soorten databases, maar welke moet men gebruiken?
  Welke is het meest geschikt bij een bepaalde soort applicatie? De keuze van de database kan een grote invloed hebben op verschillende eigenschappen: performantie, opstartsnelheid, CPU-gebruik,.. Als de database deze eigenschappen op een negatieve manier be\"invloedt, kan dit tot gevolg hebben dat het aantal gebruikers van de mobiele applicatie zal verminderen. Ter beantwoording van de probleemstelling zijn volgende deelvragen geformuleerd met betrekking op de applicatie:

  \begin{itemize}
    \item 	Wat is de invloed van de gekozen database op de opstartsnelheid? Vertraagt het gebruik van de gekozen database de opstartsnelheid van de applicatie, of heeft het helemaal geen invloed (in vergelijking met gebruik van andere databases)?
    \item Wat is de invloed van de gekozen database op het CPU-gebruik? Een hoger
    CPU-gebruik zal zorgen voor meer batterijverbruik. Zal de applicatie bij gebruik van de gekozen database meer of juist minder CPU gebruiken (in vergelijking met gebruik van andere databases)?
    \item  Wat is de gemiddelde snelheid van de gekozen database bij het toevoegen van records aan de database?
  \end{itemize}
  
  Het onderzoek werd uitgevoerd op drie verschillende applicatieprofielen: weinig data (profiel 1), gemiddelde hoeveelheid data (profiel 2), veel data (profiel 3). De verwachtingen waren dat Realm altijd de beste keuze zou zijn, behalve bij applicatieprofiel 1. Daar zou SharedPreferences de beste keuze moeten zijn, aangezien het speciaal ontwikkeld is voor kleine hoeveelheden simpele data. Het onderzoek heeft echter volgend resultaat opgeleverd:
  
  \begin{enumerate}
    \item Weinig data : Realm
    \item Gemiddelde hoeveelheid data : Realm
    \item Veel data : SQLite
  \end{enumerate}

  De details van het onderzoek zijn te vinden in het volgende deel van de scriptie.

\end{displayquote}

We gaan dit onderzoek eens onder de loep nemen, kijken wat er goed aan was en wat de eventuele verbeterpunten kunnen zijn. 

\begin{exercise}
  Probeer volgende vragen zo goed mogelijk te beantwoorden.
  \begin{enumerate}
    \item Wat is de doelstelling van het onderzoek?
    \item Wie is het publiek?
    \item Worden de conclusies expliciet gemaakt? 
    \item Schets kort hoe de structuur van het document in elkaar zit. Komt dit overeen met wat er gezien is in de les?
  \end{enumerate}
\end{exercise}

\begin{exercise}
  Duid in de abstract de hieronder opgesomde componenten aan:
  
  \begin{itemize}
    \item Context
    \item Nood
    \item Taak
    \item Object
    \item Resultaat
    \item Conclusie
    \item Perspectief
  \end{itemize}
  
  Indien je deze niet terugvindt in de tekst, probeer dan zelf een antwoord te formuleren indien jij dit onderzoek zou uitgevoerd hebben. 
\end{exercise}

\subsection{Basisconcepten in onderzoek}


\begin{exercise}[Retrieval practice: meetniveaus]
  \label{ex:retrieval-practice-meetniveaus}
  De meetniveaus zijn een belangrijk begrip in de beschrijvende statistiek omdat de meeste technieken voor visualisatie en analyse hier van afhangen. Het is dus belangrijk de verschillende meetniveaus te kennen en te begrijpen.
  
  \emph{Retrieval practice} is een studietechniek waarvan wetenschappelijk is aangetoond dat die effectief werkt en tot betere leerresultaten leidt~\parencite{RoedigerKarpicke2006}.
  
  \begin{enumerate}[label=\alph*.]
    \item Neem een leeg blad papier en probeer zonder de cursus of andere bronnen te raadplegen een overzicht te reproduceren van alle meetniveaus. Beschrijf voor elk meetniveau de specifieke eigenschappen en geef enkele voorbeelden.
    
    \textbf{Neem hier voldoende tijd voor} (bv. minstens 5 a 10 minuten). Ga niet meteen in de cursus kijken, maar probeer je zoveel mogelijk te herinneren.
    
    Wanneer je hiermee klaar bent, duid je alles wat je tot nu toe genoteerd hebt aan met een markeerstift (bv. groen).
    
    \item Overleg indien mogelijk met een medestudent en probeer samen het overzicht dat je elk gemaakt hebt te vervolledigen. Duid alle toevoegingen aan met een markeerstift in een andere kleur (bv. geel).
    
    \item Ga tenslotte in de cursus nalezen en verbeter/vervolledig zo nodig foute of ontbrekende informatie. Duid dit aan in een derde kleur (bv. oranje of rood).
  \end{enumerate}
  
  Dankzij de gemarkeerde kleuren in je nota's heb je nu een zicht op wat je al kent en waar je nog moet op studeren. Herhaal deze oefening nog een aantal keer in de loop van het semester. Kijk nooit eerst naar het resultaat van een vorige poging, maar probeer meteen op een leeg blad zoveel mogelijk informatie vanuit je geheugen terug te halen. Vergelijk achteraf wel de resultaten. Je zou moeten merken dat je de volgende keren meer en meer in het groen hebt gemarkeerd en minder of zelfs geen rood meer hebt moeten gebruiken.
\end{exercise}

\begin{exercise}
  Importeer de dataset van het onderzoek van~\textcite{Akin2016} in R. Je vindt deze terug in de Github repository voor deze cursus, onder de directory \texttt{oefeningen/datasets/} in het bestand \texttt{android\_persistence\_cpu.csv}.
  
  \begin{enumerate}
    \item Som de verschillende variabelen op in deze dataset
    \item Bepaal voor elke variabele het meetniveau
  \end{enumerate}
\end{exercise}

\begin{exercise}
  Importeer de dataset \texttt{Aardbevingen.csv}. Wat is het meetniveau van volgende variabelen?
  
  \begin{enumerate}
    \item Latitude en Longitude
    \item Type
    \item Time
    \item Depth
  \end{enumerate}

  Merk op dat in deze dataset, de kolom ID niet echt als een variabele kan beschouwd worden. Deze dient enkel om een unieke naam te geven aan elke observatie in de steekproef, maar bevat geen eigenschap.
\end{exercise}

\subsection{Steekproefmethoden en steekproeffouten}

\begin{exercise}
  Een onderzoeker wil zo correct mogelijk de consumptiegewoontes van de inwoners van 18 jaar en ouder in een bepaalde gemeente met 3 woonkernen onderzoeken.  Hij onderscheidt 4 leeftijdsgroepen zodat hij uiteindelijk aan 12 deelgroepen komt. Hij vraagt de procentuele samenstelling van de bevolking op in de gemeente en berekent daaruit hoeveel bevragingen hij per deelgroep moet uitvoeren. Dit noemen we een \emph{quotasteekproef}.
  
  Vragen:
  \begin{enumerate}[label=\alph*.]
    \item Is dit een aselecte steekproef? Waarom (niet)?
    \item Is de steekproef representatief voor de populatie?
    \item Welke soort fouten kunnen hier gemaakt worden?
    \item Wat zijn de voor- en nadelen?
    \item Welke andere parameters zouden kunnen gebruikt worden bij het opsplitsen in deelgroepen?
  \end{enumerate}
\end{exercise}

\begin{exercise}
  Een onderzoeksbureau wil het aankoopgedrag van wasproducten nagaan. Men beslist een aantal vragen te stellen aan vrouwen tussen de 25 en 55 jaar omdat men ervan uitgaat dat de relevante populatie uit deze categorie consumenten bestaat.
  
  \begin{enumerate}[label=\alph*.]
    \item Is dit een aselecte steekproef? Waarom (niet)?
    \item Welke fout wordt hier gemaakt?
    \item Hoe groot is de impact van deze fout?
  \end{enumerate}
\end{exercise}

\begin{exercise}
  De vakbonden willen een onderzoek doen naar de werkomstandigheden van de werknemers van een IT-bedrijf. Dat bedrijf heeft in totaal 3200 werknemers die verdeeld zijn over 12 vestigingen. Omdat het aantal werknemers groot is worden aselect 40 werknemers gekozen per vestiging. De steekproefomvang is dus $n = 480$.
  
  \begin{enumerate}[label=\alph*.]
    \item Is dit een aselecte steekproef? Waarom (niet)?
    \item Welke fout wordt hier gemaakt?
    \item Wat is de impact van deze fout?
    \item In welke situatie zou deze steekproefmethode toch representatief kunnen zijn voor de populatie?
  \end{enumerate}
\end{exercise}

\begin{exercise}
  We willen een onderzoek voeren naar onze studenten aan de Hogeschool Gent, faculteit Bedrijf en Organisatie. Hiervoor worden de aanwezige studenten in een bepaald opleidingsonderdeel bevraagd.
  
  \begin{enumerate}[label=\alph*.]
    \item Is dit een aselecte steekproef? Waarom (niet)?
    \item Welke fout wordt hier gemaakt?
    \item Stel dat de aanwezige docent in de perceptie van de studenten zeer streng is en tijdens de bevraging rondloopt. Welk bezwaar kan hier gegeven worden? Meer bepaald, welke fout kan op deze manier ge\"introduceerd worden?
    \item Stel dat de bevraging niet tijdens een les, maar na een examen gehouden wordt. Welke fout kan er dan gemaakt worden?
  \end{enumerate}
\end{exercise}

\begin{exercise}[Retrieval practice: steekproeffouten]
  Gebruik de procedure voor retrieval practice uit oefening~\ref{ex:retrieval-practice-meetniveaus} om de soorten \emph{steekproeffouten} in te studeren.
  
  Geef een overzicht van de verschillende soorten steekproeffouten, beschrijf ze en geef telkens een voorbeeld.
\end{exercise}